\chapter{Evaluation and Simulation Experiments}\label{sec:exp}
A number of simulation experiments has been performed to evaluate the proposed 
methods for modelling plume distributions. First, the most suitable covariance 
function and hyper-parameters were obtained. These results were then used to 
compare the different acquisition functions based on the noise-free scenarios.  
Subsequently, the performance of PDUCB was evaluated on the noisy scenarios and 
using multiple UAVs.

\section{Best Covariance Function for Plume Modelling}\label{sec:bestkernel}
To obtain the best covariance function including its parameters to approximate 
a plume distribution these were evaluated using the test-set method. For each of 
the single source Gaussian (G-NF-SS-SV), the single source dispersion 
(D-NF-SS-SV), and the multiple source dispersion (D-NF-MS-SV), all without 
noise, 50 random instances were created. For each instance a set sampling 
locations was generated using the Metropolis-Hastings based technique described 
in Chapter~\ref{sec:mh}. Herein, every fifth Metropolis-Hastings sample was used 
in the final set and was used as mean of Gaussian with a standard deviation 
$\sigma = \SI{6}{\meter}$ to draw five more samples to include in the final set.  
The proposal distribution of the Metropolis-Hastings algorithm was also 
a Gaussian with standard deviation $\sigma = \SI{6}{\meter}$. In addition, 1000 
uniformly samples were added to the final set of samples. All samples outside of 
the scenario volume were dismissed. From all kept sample points 1000 were 
randomly selected for training and the rest was used as test set to determine 
the error.

Obtaining the training samples in this way should roughly mirror a good sampling 
with an UAV with many samples in the areas of high concentration and a few in 
the remaining areas. The advantage using this way of sampling is that it allows 
us to test different kernels independently on the exact behavior of the UAV and 
time consuming simulation of it.

The kernels tested were the squared exponential, the Mat\'ern kernel with $\nu 
= 5/2$, the Mat\'ern kernel with $\nu = 3/2$, and the exponential kernel. The 
length scales tested ranged from \SI{1}{\meter} to \SI{100}{\meter}. The process 
variance was fixed as $\sigma\ped{k}^2 = 1$. Note that this parameter has no 
effect on the predictive mean as long as the assumed noise variance 
$\sigma\ped{n}^2$ is zero.

The average of the fraction of the remaining error is plotted in 
Figure~\ref{fig:lengthscales} for the different kernels and error measures. The 
minimum is roughly the same for all kernels and lies around $\ell 
= \SI{5}{\meter}$.  However, the behavior differs considerably for non-optimal 
length scales.  The smoother (the more often the kernel is differentiable) the 
more the error increases for too large length scales.  Especially, for the 
squared exponential covariance function this increase is quite abrupt. Only for 
very large length scales it decreases again for the squared exponential kernel.

\begin{figure}
    \centering
    \includegraphics{plots/lengthscales}
    \caption[Influence of the length scale of the covariance functions]{The 
        average remaining fraction of the inital error for different covariance 
        function in dependence of length scale.  The rows correspond to the 
        RMISE, WRMISE, and QRSim reward error measures.  The columns correspond 
        to a single source Gaussian (G-NF-SS), a single source Gaussian 
        dispersion (D-NF-SS), and a multiple source Gaussian dispersion 
        (D-NF-MS). All scenarios were simulated without sensor noise.  Error 
        bars represent the standard error. The boundary of $1.0$ where the error 
        of the trained Gaussian process is larger than an all zero prediction is 
        marked with a horizontal line.}\label{fig:lengthscales}
\end{figure}

Comparing the WRMISE to the RMISE the former one stays quite low even for larger 
length scales. This indicates that in the area of the plume (also due to the more 
dense sampling) a good fit is still obtained, but around that area the 
prediction gets worse. Thus, the steep concentration gradients around the plume 
are not well captured in that case.

The results give also an idea how good of a fit can be expected at best when 
using an UAV\@. Whereas the fraction of the remaining error decreases to nearly 
zero for the single source Gaussian, it stays above 0.6 for the dispersion 
scenario with the more localized plume distribution. The reward error measure is 
decreased to lower levels, but this is likely to underestimation of the error at 
the plume boundaries as argued in Chapter~\ref{sec:qrsim-reward}.

Besides the error measures the log likelihood of each trained Gaussian process 
was calculated. In Figure~\ref{fig:loglikelihood} the average over trials is 
plotted. Only for the squared exponential kernel the maximum of the log 
likelihood corresponds to the minimum of the RMISE\@. Towards longer length 
scales the likelihood declines very steeply. Using the log likelihood to 
estimate the length scales for the other covariance functions would largely 
overestimate it.

\begin{figure}
    \centering
    \includegraphics{plots/loglikelihood}
    \caption[Log likelihood in dependence of the kernel lengthscale]{The average 
        log likelihood of the training data in dependence of the length scale 
        using different kernels.  Each of the three plots shows one noise free 
        scenario of the single source Gaussian (G-NF-SS), single source Gaussian 
        dispersion (D-NF-SS), and multiple source Gaussian dispersion (D-NF-MS).  
        Error bars represent the standard error.}\label{fig:loglikelihood}
\end{figure}

Taken these results together it is best to choose a non-smooth kernel with 
a length scale of $\ell = \SI{5}{\meter}$. As it is advantageous to be able to 
use a gradient based optimizer for the optimization of acquisition functions, 
I decided to use the Matérn kernel with $\nu = 3/2$ in the further experiments, 
which gives a once mean square differentiable Gaussian process in opposite to 
the exponential kernel. Unfortunately, optimizing the length scale using the 
likelihood would not give good results and I fixed the length scale at $\ell 
= \SI{5}{\meter}$. Also, including a prior in the log likelihood does not help 
here. In example to shift the maximum of the likelihood for the chosen kernel to 
\SI{5}{\meter} a Gaussian prior would need a standard deviation of less then 
$\sigma_{\ell} < \e^{-2078} / \sqrt{2\uppi} \approx 0$ (see 
Apendix~\ref{sec:prior}).  Thus, effectively resulting in a fixed length scale.

\section{Comparison of Utility Functions}\label{sec:cmputility}
Given the kernel chosen in the previous section I continued to compare the 
different utility functions in the noiseless scenarios single source Gaussian 
(G-NF-SS-SV), single source dispersion (D-NF-SS-SV), and multiple source 
dispersion (D-NF-MS-SV).

For each given scenario 20~trials were performed. In each run the UAV first 
surrounded the simulation area in a height of \SI{40}{\meter} with a margin of 
\SI{10}{\meter} to the boundaries of the simulated volume. After that further 
way-points were chosen with one of the acquisition functions discussed in 
Chapter~\ref{sec:utility}. The optimization of that functions has been described 
in Chapter~\ref{sec:fnopt}.

Each trial was allowed to run for a maximum of \SI{3000}{\second} in simulation 
time. However, when a new target way-point was within \SI{3}{\meter} of the 
previous one the UAV was considered to become stuck in a maximum of the 
acquisition function and the simulation was stopped at that point to reduce 
overall simulation time. A plume measurement was taken every second.

The error measures were estimated as described in Chapter~\ref{sec:error}. The 
sampling locations for that where chosen as 1000 uniformly distributed sampling 
locations, every tenth of 4200 locations from the Metropolis-Hastings algorithm 
with Gaussian proposal distribution with standard deviation $\sigma 
= \SI{10}{\meter}$, and 10 more locations sampled from the proposal distribution 
for each of included Metropolis-Hastings samples.

I tested all three utility functions proposed in Chapter~\ref{sec:utility}: 
DUCB, PDUCB (with $\varepsilon = 10^{-30}$), and GO\@. DUCB was tested with 
a constant scaling factor of $s\ped{DUCB}(\vc y) = 1$ and the automatic scaling 
in Equation~\ref{eqn:scale-ducb}.  PDUCB was tested with a constant scaling 
factor of $s\ped{PDUCB}(\vc y) = 70$ (a little bit more than $-\log 
\varepsilon$) and the automatic scaling in Equation~\ref{eqn:scale-pducb}.  
Furthermore, I performed a parameter search over $\kappa \in \cbr{0.1, 0.5, 
0.75, 1, 1.25, 1.5, 2}$ and $\gamma \in \cbr{0} \cup \cbr{-10^p | p = -9, -8, 
  \dots, -2}$. Note that for the GO utility function the $\kappa$ parameter has 
no effect.

Figure~\ref{fig:psearch-G-NF-SS-SV}--\ref{fig:psearch-D-NF-MS-SV} visualize the 
normalized error for the different scenarios.  The respective parameters and 
values of the minima (excluding the QRSim reward) are listed in 
Table~\ref{tbl:err-g-nf-ss-sv}--\ref{tbl:err-d-nf-ms-sv}.  The average reduction 
(over trials) of the RMISE against simulation time in the single source Gaussian 
scenario (G-NF-SS-SV) is plotted in Figure~\ref{fig:errtrace-nf} and looks 
essentially the same for WRMISE and the QRSim reward and therefore it is not 
shown fer those measures. Finally, Figure~TODO visualizes some example UAV 
trajectories for the different acquisition functions.

\begin{figure}
    \centering
    \includegraphics{plots/psearch-G-NF-SS-SV}
    \caption[Remaining fraction of the initial error (G-NF-SS-SV)]{The average 
        remaining fraction of the initial error for different measures, utility 
        functions, parameters in the noiseless single source Gaussian scenario 
        (G-NF-SS-SV).  The columns represent the RMISE, WRMISE, and QRSim reward 
        error measure.  The rows represent the DUCB, auto-scaled DUCB, PDUCB, 
        auto-scaled PDUCB and GO utility functions. The auto-scaled versions use 
        the scaling factor defined in Equations~\ref{eqn:scale-ducb} 
        and~\ref{eqn:scale-pducb}, in contrast to a constant scaling factor. The 
        minimum of each plot is marked with 
        cross.}\label{fig:psearch-G-NF-SS-SV}
\end{figure}

\begin{figure}
    \centering
    \includegraphics{plots/psearch-D-NF-SS-SV}
    \caption[Remaining fraction of the initial error (D-NF-SS-SV)]{The average 
        remaining fraction of the initial error for different measures, utility 
        functions, parameters in the noiseless single source Gaussian dispersion 
        scenario (D-NF-SS-SV).  The columns represent the RMISE, WRMISE, and 
        QRSim reward error measure.  The rows represent the DUCB, auto-scaled 
        DUCB, PDUCB, auto-scaled PDUCB and GO utility functions. The auto-scaled 
        versions use the scaling factor defined in 
        Equations~\ref{eqn:scale-ducb} and~\ref{eqn:scale-pducb}, in contrast to 
        a constant scaling factor.  The minimum of each plot is marked with 
        cross.}\label{fig:psearch-D-NF-SS-SV}
\end{figure}

\begin{figure}
    \centering
    \includegraphics{plots/psearch-D-NF-MS-SV}
    \caption[Remaining fraction of the initial error (D-NF-MS-SV)]{The average 
        remaining fraction of the initial error for different measures, utility 
        functions, parameters in the noiseless multiple source Gaussian 
        dispersion scenario (D-NF-SS-SV).  The columns represent the RMISE, 
        WRMISE, and QRSim reward error measure.  The rows represent the DUCB, 
        auto-scaled DUCB, PDUCB, auto-scaled PDUCB and GO utility functions. The 
        auto-scaled versions use the scaling factor defined in 
        Equations~\ref{eqn:scale-ducb} and~\ref{eqn:scale-pducb}, in contrast to 
        a constant scaling factor.  The minimum of each plot is marked with 
        cross.}\label{fig:psearch-D-NF-MS-SV}
\end{figure}

\newenvironment{errtbl}{\begin{tabular}{lllSSSS}\toprule}{\bottomrule\end{tabular}}
\newcommand*{\errtblhead}[1]{
        & & &
        \multicolumn{2}{c}{#1} &
        \multicolumn{2}{c}{Norm.\ #1} \\
        \cmidrule(lr){4-5} \cmidrule(lr){6-7}

        Utility function &
        \multicolumn{1}{l}{$\kappa$} &
        \multicolumn{1}{l}{$\gamma$} &
        \multicolumn{1}{c}{Mean} &
        \multicolumn{1}{c}{Std} &
        \multicolumn{1}{c}{Mean} &
        \multicolumn{1}{c}{Std} \\
        & & &
        \multicolumn{1}{c}{\si{\nano\gram\per\meter\cubed}} &
        \multicolumn{1}{c}{\si{\nano\gram\per\meter\cubed}} &
        & \\ \midrule }

\begin{table}
    \centering
    \begin{errtbl}
        \errtblhead{RMISE}
        DUCB & 1.25 & \num{-1e-07} & 249.47 & 154.87 & 0.29 & 0.20 \\
        auto-scaled DUCB & 1.50 & \num{-1e-08} & 250.56 & 154.55 & 0.31 & 0.25 
        \\
        PDUCB & 0.10 & \num{-1e-09} & 191.16 & 198.60 & 0.18 & 0.13 \\
        auto-scaled PDUCB & 0.10 & \num{-1e-07} & 190.73 & 180.43 & 0.18 & 0.12 \\
        GO & 0.10 & \num{-1e-08} & 745.74 & 317.59 & 0.80 & 0.13 \\
        \midrule
        \\
        \errtblhead{WRMISE}
        DUCB & 1.25 & \num{-1e-07} & 115.25 & 109.04 & 0.21 & 0.22 \\
        auto-scaled DUCB & 1.50 & \num{-1e-08} & 122.45 & 112.15 & 0.25 & 0.28 
        \\
        PDUCB & 0.10 & \num{-1e-06} & 91.73 & 122.39 & 0.13 & 0.14 \\
        auto-scaled PDUCB & 0.10 & \num{-1e-07} & 92.62 & 120.08 & 0.13 & 0.13 \\
        GO & 0.10 & \num{-1e-08} & 472.51 & 219.39 & 0.80 & 0.15 \\
    \end{errtbl}
    \caption[Minimal error values G-NF-SS-SV.]{The minimal obtained error (RMISE 
        and WRMISE) for each acquisition function and the parameter values used 
        in the single source Gaussian scenario 
        (G-NF-SS-SV).}\label{tbl:err-g-nf-ss-sv}
\end{table}

\begin{table}
    \centering
    \begin{errtbl}
        \errtblhead{RMISE}
        DUCB & 2.00 & \num{0.0} & 5.33 & 3.90 & 0.94 & 0.10 \\
        auto-scaled DUCB & 1.50 & \num{-0.0001} & 5.00 & 3.67 & 0.91 & 0.16 \\
        PDUCB & 0.50 & \num{-1e-09} & 4.19 & 2.99 & 0.75 & 0.24 \\
        auto-scaled PDUCB & 1.00 & \num{-1e-09} & 4.10 & 2.85 & 0.76 & 0.24 \\
        GO & 0.10 & \num{0.0} & 5.40 & 3.88 & 0.95 & 0.09 \\
        \midrule
        \\
        \errtblhead{WRMISE}
        DUCB & 2.00 & \num{0.0} & 3.48 & 3.04 & 0.91 & 0.22 \\
        auto-scaled DUCB & 1.50 & \num{-0.0001} & 3.28 & 2.95 & 0.88 & 0.26 \\
        PDUCB & 0.50 & \num{-1e-05} & 2.29 & 2.20 & 0.63 & 0.39 \\
        auto-scaled PDUCB & 1.00 & \num{-1e-07} & 2.37 & 2.33 & 0.64 & 0.37 \\
        GO & 0.10 & \num{0.0} & 3.62 & 3.00 & 0.94 & 0.17 \\
    \end{errtbl}
    \caption[Minimal error values D-NF-SS-SV.]{The minimal obtained error (RMISE 
        and WRMISE) for each acquisition function and the parameter values used 
        in the single source Gaussian dispersion scenario 
        (D-NF-SS-SV).}\label{tbl:err-d-nf-ss-sv}
\end{table}

\begin{table}
    \centering
    \begin{errtbl}
        \errtblhead{RMISE}
        DUCB & 1.25 & \num{0.0} & 20.63 & 11.60 & 0.93 & 0.08 \\
        auto-scaled DUCB & 2.00 & \num{-0.0001} & 18.45 & 10.97 & 0.83 & 0.13 \\
        PDUCB & 0.50 & \num{0.0} & 16.55 & 11.76 & 0.68 & 0.18 \\
        auto-scaled PDUCB & 1.25 & \num{-1e-06} & 16.62 & 11.87 & 0.68 & 0.19 \\
        GO & 0.10 & \num{0.0} & 20.70 & 11.76 & 0.92 & 0.11 \\
        \midrule
        \\
        \errtblhead{WRMISE}
        DUCB & 0.50 & \num{-1e-05} & 14.11 & 10.03 & 0.93 & 0.14 \\
        auto-scaled DUCB & 2.00 & \num{-0.0001} & 12.11 & 9.60 & 0.80 & 0.21 \\
        PDUCB & 1.00 & \num{-1e-06} & 10.99 & 10.86 & 0.62 & 0.29 \\
        auto-scaled PDUCB & 1.25 & \num{-1e-05} & 10.80 & 10.59 & 0.65 & 0.26 \\
        GO & 0.10 & \num{0.0} & 14.06 & 9.96 & 0.93 & 0.12 \\
    \end{errtbl}
    \caption[Minimal error values D-NF-MS-SV.]{The minimal obtained error (RMISE 
        and WRMISE) for each acquisition function and the parameter values used 
        in the multiple source Gaussian dispersion scenario 
        (D-NF-MS-SV).}\label{tbl:err-d-nf-ms-sv}
\end{table}

\begin{figure}
    \centering
    \includegraphics{plots/errtrace-nf}
    \caption[Time-course of the error reduction]{The average remaining fraction 
        of the initial RMISE in the single source Gaussian scenario 
        (G-NF-SS-SV).  Each individual plot corresponds to one utility function 
        and scaling.  The auto-scaled versions use the scaling factor defined in 
        Equations~\ref{eqn:scale-ducb} and~\ref{eqn:scale-pducb}, in contrast to 
        a constant scaling factor. The $\gamma$ parameter is coded by hue and 
        the $\kappa$ parameter by lightness.}\label{fig:errtrace-nf}
\end{figure}

This is a rich dataset from which quite a few insights can be gained. First of 
all it can be noted that the GO acquisition function does not perform very well.  
Even in the single source Gaussian scenario the RMISE is only reduced by about 
\SI{20}{\percent} and in the other two scenarios it performs even worse.

Comparing DUCB and PDUCB the latter one consistently performs better with 
a reduction in the RMISE and WRMISE by at least additional \SI{11}{\percent} and 
in the dispersion scenarios even more.  Despite that, the standard deviation of 
PDUCB is higher almost always higher than that of DUCB\@.

In the single source Gaussian scenario PDUCB proves to be quite robust against 
the choice of $\kappa$ as for all values very good results are obtained. This 
picture is a bit more noisy in the dispersion scenarios. It seems that too low 
values ($\kappa < 0.5$) degrade performance. This is consistent with the 
argument in Chapter~\ref{sec:utility} that a too low $\kappa$ limits the 
exploration and let the UAV become stuck in a (local) maximum. The choice of 
$\gamma$ has no considerable effect as long as the distance penalty is not 
chosen too large ($\gamma < -10^{-5}$).

The same behavior for the choice of $\gamma$ is also observed for DUCB in the 
single source Gaussian scenario. However, this utility function is far more 
sensitive to the choice of $\kappa$. Using the scaling $s\ped{DUCB}(\vc y) = 1$ 
the performance degrades setting $\kappa < 1$ and using the automatic scaling it 
degrades for $\kappa < 1.5$. In the dispersion scenarios the DUCB acquisition 
function does not perform well for any tested combination of parameter values.

Interestingly,  DUCB performs slightly better with the automatic scaling, 
whereas PDUCB performs slightly worse.

Finally, taking a look at the time course of error reduction multiple phases can 
be discovered where a reasonable reduction of the error occurs. About the first 
\SI{500}{\second} nearly no reduction occurs as in this phase the UAV only 
surround the area of interest. Then the error rapidly decreases in the next 
\SIrange{1000}{2000}{\second} until the decrease levels off and stays fairly 
constant for the rest of the simulation time.

These results show that PDUCB outperforms the DUCB and GO acquisition functions 
and in addition is quite robust against a non-optimal choice of parameters. The 
especially bad performance of the GO utility function is not too surprising at 
its intended use is to find a function maximum and not building a correct model 
of the function (respectively plume concentration).  DUCB works reasonable well 
for the simple case of a Gaussian distribution, but fails for the more localized 
dispersions.

The PDUCB performance might not seem to be too impressive in the dispersion 
scenarios, too. However, one has to keep in mind that even in 
Section~\ref{sec:bestkernel} with much more samples the RMISE could not be 
reduced to less than \SI{61}{\percent}.  Also, the qualitatively the plume is 
predicted at the correct location as Figure~TODO shows, despite some deviance of 
the exact concentration values.

Given this discussion I decided to limit further experiments to the PDUCB 
acquisition function as the other options seem to not a viable option (except 
maybe for the single source Gaussian). As smaller values of $\kappa$ seem to 
provide slightly better results, but it should not be below $1$ as argued, 
a value of $\kappa = 1.25$ was selected. The penalty distance seems to slightly 
improve the results up to \num{-1e-5}. However, to prevent to fall in the area 
where the performs then quickly decreases, it was set to a lower value of 
$\gamma = \num{-1e-7}$ which also good results. Both scaling approaches 
(constant or automatic) where used in further experiments as it is not clear 
from these results which one is better. Though, the automatic scaling is a bit 
worse in terms of error, it has one less parameter which would be set according 
to the range of concentration values.

TODO example run visualizations

\section{Evaluation in a Noisy Setting}\label{sec:noisy}
It is important to verify that the methods for plume modelling also work in 
a noisy environment as in the real world noise will necessarily occur. For that 
a standard deviation of the sensor noise of $\sigma\ped{noise} 
= \SI{1e-5}{\gram\per\meter\cubed}$ was assumed. As it should be possible to 
reliably estimate the noise level of the sensor, I considered this value as 
known. That allows to set the noise variance of the Gaussian process 
$\sigma\ped{n}^2 = \sigma\ped{noise}^2$. For a good prediction the ratio of 
$\sigma\ped{n}^2$ and the kernel variance $\sigma\ped{k}^2$ has to be chosen 
well. The process of doing that will be discussed in the next section before 
returning to the actual simulations.

\subsection{Choosing the Kernel Variance}
The method for determining the kernel variance was essentially the same as 
described in Section~\ref{sec:bestkernel} for determining the length scale. The 
points were it differs are that instead of the noiseless scenarios the single 
and multiple source plume dispersion scenarios with sensor noise (D-SN-SS-SV, 
D-SN-MS-SV) were used and only the Mat\'ern kernel with $\nu = 3/2$ was used.  
Instead of varying the length scale it was fixed to $\ell = \SI{5}{\meter}$ and 
the kernel variance $\sigma\ped{k}^2$ was varied from 
\SIrange{1e-12}{1e-3}{\gram\squared\per\meter\tothe{6}}.

The results are shown in Figure~TODO\@. The normalized error measures are 
highest for low values of $\sigma\ped{k}^2$ approaching 1. For $\sigma\ped{k}^2 
> \SI{1e-9}{\gram\squared\per\meter\tothe{6}}$ the WRMISE stays the same, but 
the RMISE and QRSim reward slightly increase. This is even less pronounced for 
multiple sources.

As all error measures do have their minimum at $\sigma\ped{k}^2 
= \SI{1e-9}{\gram\squared\per\meter\tothe{6}}$ or have almost reached it, this 
value has been used in the following.

\subsection{Simulation of the Scenarios Including Noise}
Using the kernel variance of $\sigma\ped{k}^2 
= \SI{1e-9}{\gram\squared\per\meter\tothe{6}}$ determined in the previous 
section the PDUCB method was evaluated in the single and multiple source plume 
dispersion scenario including sensor noise (D-SN-SS-SV, D-SN-MS-SV). The sensor 
noise variance and the noise variance of the Gaussian process were set to 
$\sigma\ped{noise}^2 = \sigma\ped{n}^2 
= \SI{1e-10}{\gram\squared\per\meter\tothe{6}}$.  As the change of 
$\sigma\ped{k}^2$ (previously set to 1) influences $\sigma^2(\vc x)$ the value 
of $\kappa$ has to be adjusted by the inverse factor. Hence, the PDUCB parameter 
setting in the following were $\kappa = \num{1.25e9}$ and $\gamma 
= \num{-1e-7}$.

In many instances surrounding the simulation area in just one height is not 
sufficient to discover the plume under the influence of noise. Thus, the 
complete search and the wind based search strategy described in 
Chapter~\ref{sec:bootstrapping} have been employed. The latter approach requires 
of course wind information which was read out from the QRSim simulator.

Apart from these points the same methods as in the noiseless case 
(Section~\ref{sec:cmputility}) were used including the same number of 20~trials.

The results for a single source are summarized in Figure~TODO and Table~TODO\@.  
All tested variants reduce the normalized RMISE to approximately \num{0.75} with 
a similar standard deviation around \num{0.23}. Using the wind based search the 
error starts to decrease earlier as less area has to be covered. Also the 
overall decrease seems to be a bit faster after the plume has bee discovered.

Looking at the WRMISE it turns out that the wind based search decreases the 
normalized error by about \SI{7}{\percent} additionally compared to the complete 
search. Thus, the wind based search is able to better approximate the actual 
plume without increasing the approximation error in other area.

With regard to the WRMISE the automatic scaling seems to perform a bit better 
(difference of \num{0.05} in the normalized error), but this is not the case for 
the RMISE\@.

TODO multiple sources
TODO discussion/conclusion

\section{Multiple UAVs}
Finally, the performance of the PDUCB acquisition function with the extension 
for multiples UAVs (Chapter~\ref{sec:multiple-uavs}) has been evaluated. For 
this exactly the same methods as in the previous section were used with 
exception of the scenario. This was replaced by the multiple UAV, multiple 
source plume dispersion scenario (D-SN-MS-MV). Also, a number of different $\rho 
\in TODO$ has been tested.

TODO
