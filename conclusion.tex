\chapter{Conclusion}
Gaussian processes have been used before in modelling of spatial data and 
environmental monitoring. However, previous approaches, namely DUCB, do not work 
well for plume dispersions as the simulation experiments have shown. Presumably, 
the steep concentration gradients and high locality are the main problem.

In this work DUCB has been adapted to PDUCB which could be shown to work 
reasonable well also for plume dispersions. Nevertheless, there is still room 
for improvement as the error is seldom reduced by more than TODO\SI{1}{\percent} 
on average. The PDUCB algorithm could successfully be extended to multiple UAVs 
to increase the speed of mapping the dispersion.

Besides the modelling of the plume distribution, localizing the dispersion at 
all is also an important problem. Noisy data does not allow to reliably estimate 
a concentration gradient. This requires to use a systematic search approach.  
Incorporating information of the wind direction speed up the search. Once 
a plume has been found PDUCB maps it quickly.

One problem that could not be solved in the scope of this thesis is an automatic 
selection of hyper-parameters based on the data. The usual approach of 
likelihood optimization clearly fails. Selecting hyper-parameters on a trial to 
trial basis would also allow a closer match of the prediction to the plume 
dispersion.

Besides that, the prediction quality might be improved if wind information is 
considered and included in the covariance function. Some pointer in that 
direction, though for a different algorithm, are given by 
\textcite{Reggente:2009ti}. In general non-stationary kernels could improve the 
prediction, but they come at the cost of more hyper-parameters and prior 
assumptions about the plume dispersion which might be violated.

In summary, the basic applicability of Gaussian processe with a PDUCB 
acquisition function for plume distribution modelling has been shown. However, 
there is quite a number of possibilities of improvement left for future work.  
Also, it should be shown in future work that the proposed methods work in a real 
world scenario as only simulations have been performed. Further interesting 
research directions following from this work would be the inclusion and handling 
of obstacles or the modelling of time-varying plume distributions.

