\chapter{Outlook on Time-varying Plumes}\label{sec:timevarying}
So far only distributions which are static over time have been discussed. This 
assumption might be violated in many instances. Unfortunately, time did not 
allow me to extend the proposed methods to time-varying plumes. Nevertheless, 
I provide some thoughts on how to do this.

The first thing to do is probably adding another input dimension to the Gaussian 
process representing time. This also requires adjusting the covariance function.  
As the temporal correlations might differ from the spatial correlations 
a product of a temporal and a spatial kernels would be a good start. Though that 
separability neglects potential spatio-temporal interdependencies, construction 
and hyper-parameter estimation is easier. Given wind with a strong 
directionality, a kernel modeling the spatio-temporal interdependencies becomes 
more important. Some work on separable as well as non-separable spatio-temporal 
covariance functions with application to environmental monitoring has been 
published by \textcite{Singh:2010wt}.

Diffusion and advection by wind are the two main factors leading to a change of 
the plume distribution over time. While diffusion is a rather slow process, 
advection can occur on shorter time scales. Thus, it might be especially 
beneficial to include wind effects in the covariance function.  
\Textcite{Reggente:2009ti,Lilienthal:2009ij} did this in another kernel based 
approach called Kernel DM+V/W algorithm.

A specific scenario with a time-varying plume distribution is suggested by 
\textcite{denardi2013rn}. Instead of having a constant plume dispersion like in 
the scenarios discussed, each source emits puffs in random intervals travelling 
with the wind.  Given a regular emission interval this could be modeled by using 
the value of a periodic function as time input or incorporating a periodic 
function into the covariance function.  However, given a random emission 
interval this probably does not work as the frequency of the periodic function 
would have to change.

Another problem to be solved is locating such a puff dispersion. Already for the 
static dispersion an extensive search is required in the beginning. Given a puff 
dispersion measuring a low concentration could mean that the location does not 
lie in the path of the puff dispersion, but also that the measurement was taken 
between to puffs. Unfortunately, this seems like an inherent problem which 
cannot be completely solved.

Finally, let me discuss two properties of Gaussian processes to take into 
account when modeling time-varying plume distributions. When using a zero mean 
prior (as it is usually done), the predicted concentration mean will decay to 
zero over time (assuming $k(t, t') \rightarrow 0$ for $\abs{t - t'} \rightarrow 
\infty$). However, in certain scenarios (like puff dispersions) it is likely 
that an increased concentration is measured again at a location in the future if 
this was the case once before at that location. It might be a good idea to 
adjust the mean prior based on the measurements to prevent a decay of the mean 
prediction.  Note that the covariance function remains unchanged and the 
uncertainty at that location will still increase with time indicating that the 
measurement should be repeated.

The performance in time-varying scenarios might be impaired by the property that 
Gaussian processes have no notion of directionality of time. This follows from 
the (required) symmetry of the covariance function which does not allow to 
differentiate between $t < t'$ and $t > t'$. Hence, a kernel might model the 
path along which a plume travels, but not the direction. Along that direction 
the plume is likely to broaden because of dispersion. Respectively it gets more 
concentrated in the opposite direction. Effects like this cannot be modeled 
except for the overall statistics.
