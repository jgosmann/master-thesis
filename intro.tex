\chapter{Introduction}
Environmental monitoring is used to ensure water and air pollution levels are in
compliance with governmental regulations \parencite[i.\,e.][]{Anonymous:1996ui}, 
to monitor ozone concentrations and climate change,  as well as surveillance of 
industrial facilities for leakages of pollutants to just name a few 
applications.

In many of these scenarios it is feasible to have a static sensor network.  It 
is, therefore, not surprising the research on optimal sensor placement at fixed 
locations exists \parencite[e.\,g.][]{Osborne:2008hi, Guestrin:2005cq, Wang:kz}.  
However, better results might be obtainable using mobile robots which can move 
to interesting areas and acquire more precise data there. Moreover, in some 
scenarios like disaster response, where timely information is needed, it might 
not be possible to first deploy an extensive sensor network. In this case mobile 
robots allow here to quickly identify the interesting regions. The problem of 
autonomously choosing the best locations for data acquisition is known as 
\newterm{active learning}. \textcite{Marchant:2012wb} also used the term 
\newtermAbbrev{intelligent environmental monitoring}{IEM}.

In this work, I focus on a scenario proposed as part of the CompLACS project in 
\textcite{denardi2013rn}: One or more sources emit a gaseous substance or 
aerosol which is dispersed by a constant wind. The resulting plume distribution 
has to be estimated with autonomously controlled \pabbrev{unmanned aerial 
    vehicle}{UAV}. The rather steep concentration gradients and small spatial 
extend orthogonal to the main dispersion axis add to the difficulty of this 
problem.  With a few UAVs it is not possible to cover the whole volume of 
investigation using a regular pattern densely enough in a timely manner.  It is 
necessary to focus on measurements in specific areas.  Furthermore, measurement 
noise has to be taken into consideration.

In previous works swarms of robots have been used to localize the source of 
a plume \parencite{Jatmiko:2007df, Zarzhitsky:2005tz}. These approaches, 
however, do not allow the usage of only one robot and do not provide one with an 
estimation of the overall plume distribution. Such an estimation might be 
important for various reasons such as determining areas in which a threshold is 
exceeded or a contamination occurred. As \textcite{Reggente:2009ti} noted, 
though one could try to model the actual fluid dynamics to obtain these 
information, such computational fluid dynamics models become intractable for 
real world applications with inaccurate data.  Instead they propose to build 
a statistical model with the gas concentration measurements as random variables.

A widely used statistical model for spatial or spatio-temporal data are 
\newterm{Gaussian processes}\footnote{In geospatial statistics the modelling 
    with Gaussian processes is also known as \newterm{kriging}.}. In several 
works \parencite[e.\,g.][]{Stachniss:2008vz, Marchant:2012wb} the modelled data 
were actually gas concentrations. Also, there exists some prior work on actively 
selecting the sampling locations. \Textcite{Stranders:2008wl} do this for 
discrete locations; \textcite{Singh:2010wt} and \textcite{Marchant:2012wb} for 
continuous locations. However, none of these approaches is optimal for plume 
dispersions.

TODO overview over chapters
