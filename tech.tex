\chapter{Technical Details}\label{sec:tech}
To implement the discussed algorithms in a functioning system some further 
details have to be taken into consideration. I will discuss these in the 
following after giving a short general overview of the implementation.

\section{Implementation}
All simulations were performed with QRSim \parencite{denardi2013rn}, a quadrotor 
simulator developed specifically to test high level tasks. It supports multiple 
UAVs, which are simulated with realistic platform dynamics.  Also, equipped 
sensors (e.\,g.~GPS, IMU) are simulated with different sources of inaccuracies.  
This includes wind influencing the vehicles as well as the plume dispersions.

I decided to implement the algorithms (i.\,e.~Gaussian processes, acquisition 
functions) in Python because of its high-level programming constructs and 
excellent scientific computing support with NumPy and SciPy 
\parencite{Oliphant:2007dm}. Communication between the Python part and QRSim 
implemented in MATLAB were done with a TCP interface based on the Google 
protocol 
buffers\footnote{\url{https://developers.google.com/protocol-buffers/}}.

Though there exist several Gaussian process implementations for Python like for 
example Scikit-learn \parencite[i.\,e.][]{scikit-learn}, none supports online 
updates to my knowledge.  For that reason I developed an own implementation.

\section{Function optimization}\label{sec:fnopt}
For finding the maximum of an acquisition function the SciPy wrapper of the 
\mbox{FORTRAN} implementation of the \mbox{L-BFGS-B} algorithm 
\parencite{Byrd:2006iv, Zhu:1997br} was used.  As gradient based optimizer with 
the possibility to constrain the search space to the task volume it is well 
suited for the task.

To choose the starting location $\vc x_0$ the utility function was evaluated on 
a coarse $5 \times 5 \times 5$ grid and the location with the maximal value was 
chosen. The convergence parameters were set to $\mathit{pgtol} = 10^{-10}$ and 
$\mathit{factr} = 100$. The rather flat DUCB gradient required this strict 
settings. For the other acquisition functions a good convergence was also 
possible with higher values (i.\,e.~$\mathit{pgtol} = 10^{-5}$ and $\mathit{factr} 
= 10^7$).  Nevertheless, the same parameters were used for all utility 
functions.

Noisy plume measurements produce a large number of local maxima in the utility 
function. Hence, in those scenarios the optimization has to be performed 
multiple times using some additional starting points. For this Gaussian 
distributions with a standard deviation of \SI{5}{\meter} centered on the 
current UAV position and the location of the maximal recorded plume measurement 
were used.  From each of these Gaussian five additional starting locations for 
the optimization were sampled.

\section{UAV Control}
The QRSim simulator accepts different UAV control commands including way-points 
and target velocities. Setting directly the way-points would be natural as the 
optimization of the acquisition function leads to target coordinates. However, 
the QRSim way-point controller did not proof to be very reliable and UAVs 
leaving the simulation area were not uncommon. To circumvent this, a different 
algorithm was used. Each way-point $\vc t_i$ was translated to velocity commands 
$\vc v_i$ sent to QRSim for the $i$-th out of $n$ UAVs with
\begin{align}
        \vc v_i &= \del{\begin{array}{c}
            d_{i, 1} \min\{1, v_{\max,1} / d_{i,\mathrm{hor}}\} \\
            d_{i, 2} \min\{1, v_{\max,2} / d_{i,\mathrm{hor}}\} \\
            \min\{v_{\max,3}, \max\{-v_{\max,3}, d_{i, 3}\}\}
        \end{array}} \label{eqn:final_velocities} \\
        d_{i,\mathrm{hor}} &= \sqrt{d_{i, 1}^2 + d_{i, 2}^2} \\
    \begin{split}
        \vc d_i &= \del{d_{i,1}, d_{i,2}, d_{i,3}}\Tr \\
        &= \diag\!\del{\vc v\ped{\max}} \del{u_1 \del{\vc t_i - \vc x_i} + u_2 
            \sum_{j = 1,\ i \neq j}^{n} \frac{\vc x_i - \vc x_j}{\abs{\vc x_i 
                    - \vc x_j}^3}}
    \end{split}
\end{align}
where $\vc x_i$ are the current UAV positions, $u_1 = \SI{0.025}{\per\meter}$ 
and $u_2 = \SI{5}{\meter\squared}$ are scaling constants, and $\vc v_{\max} 
= (v_{\max,1}, v_{\max,2}, v_{\max,3})\Tr = (\SI{6}{\meter\per\second}, 
\SI{6}{\meter\per\second}, \SI{6}{\meter\per\second})\Tr$ a vector of speed 
limits\footnote{QRSim additionally applies its own speed limit independently per 
    direction in the UAV body frame of reference. It is 
    \SI{3}{\meter\per\second} for the two horizontal axes.}.  The $\vc d_i$ 
vector components consist of two terms.  One directs the speed towards the 
target proportional to the remaining distance resulting in a slow down near to 
the target. The remaining term acts like a repelling force between the UAVs to 
keep them from colliding.  It is proportional to the inverse of the square of 
the distance. The power of three occurs because the direction vector $\vc x_i 
- \vc x_j$ has to be normalized.  With Equation~\ref{eqn:final_velocities} the 
velocity will be limited while keeping the overall horizontal direction.

To prevent the UAVs from going astray a safety margin of \SI{10}{\meter} is 
defined at the boundaries of the simulated volume. When an UAV enters this 
margin in one dimension the corresponding velocity component will be set to $\pm 
\vc v_{\max}$.

A target $\vc t_i$ is considered to be reached when $\abs{\vc t_i - \vc x_i} 
< \SI{3}{\meter}$.
