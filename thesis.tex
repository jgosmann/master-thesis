\documentclass[11pt,a4paper,twoside,BCOR=15mm,listof=totoc]{scrbook}

%\usepackage[utf8]{inputenc}
%\usepackage[T1]{fontenc}
%\usepackage{scrpage2}
\usepackage[tworuled]{algorithm2e}
\usepackage{amssymb}
\usepackage{amsmath}
\usepackage{amsthm}
\usepackage[style=authoryear,backend=biber,language=american]{biblatex}
\usepackage{bm}
\usepackage{booktabs}
\usepackage{commath}
\usepackage{dcolumn}
\usepackage{enumerate}
%\usepackage[flushmargin,ragged]{footmisc}
\usepackage{gitinfo}
\usepackage{graphicx}
%\usepackage{ifthen}
%\usepackage{hhline}
%\usepackage{lastpage}
%\usepackage{listings}
\usepackage{nag}
\usepackage[refpage,intoc]{nomencl}
\usepackage{longtable}
\usepackage{polyglossia}
%\usepackage{ngerman}
\usepackage[retain-unity-mantissa=false,exponent-product=\cdot]{siunitx}
%\usepackage{scrpage2}
\usepackage{setspace}
\usepackage{subfigure}
\usepackage{upgreek}
%\usepackage{xcolor}
%\usepackage{xr-hyper}
\usepackage{hyperref} % should be loaded last
\usepackage[acronym,nogroupskip,nonumberlist,toc]{glossaries} % needs to be loaded after hyperref

\newglossarystyle{lltable}{%
    \renewenvironment{theglossary}{\begin{longtable}[l]{lp{10cm}}}{\end{longtable}}%
    \renewcommand*{\glossaryheader}{}%
    \renewcommand*{\glsgroupheading}[1]{}%
    \renewcommand*{\glsgroupskip}{}%
    \renewcommand*{\glossentry}[2]{\glossentryname{##1}&\raggedright\glossentrydesc{##1}\tabularnewline}%
    \renewcommand*{\subglossentry}[3]{\glossentry{##2}{##3}}%
}
\setglossarystyle{lltable}
\makeglossaries{}

\setdefaultlanguage{english}
\setotherlanguages{german}

\newcolumntype{d}{D{.}{.}{2}}

\newtheorem{lemma}{Lemma}
\newtheorem{theorem}{Theorem}
\newtheorem{corollary}{Corollary}

%\lstset{basicstyle=\fontfamily{pcr}\selectfont, keywordstyle=\bfseries, language=C++}
\sisetup{per-mode=symbol}

%\pagestyle{scrheadings}

\newcommand{\vc}[1]{\bm{#1}}
\newcommand{\vcc}[1]{\textbf{#1}}
\newcommand{\mat}[1]{\bm{#1}}
\newcommand{\Tr}{^{\top}}
\newcommand{\e}{\mathrm{e}}
\DeclareMathOperator*{\argmax}{argmax}
\DeclareMathOperator{\cov}{cov}
\DeclareMathOperator{\diag}{diag}
\DeclareMathOperator*{\mslim}{mslim}
\DeclareMathOperator{\mean}{mean}
\DeclareMathOperator{\std}{std}

\newcommand{\ped}[1]{_{\mathrm{#1}}}

\newcommand{\newterm}[1]{\emph{#1}}
\newcommand{\abbrev}[2]{#1 (#2\newacronym{#2}{#2}{#1})\glsadd{#2}}
\newcommand{\pabbrev}[2]{#1s (#2\newacronym{#2}{#2}{#1})\glsadd{#2}}
\newcommand{\newtermAbbrev}[2]{\newterm{#1} (#2\newacronym{#2}{#2}{#1})\glsadd{#2}}

\newacronym{GNFSSSV}{G-NF-SS-SV}{Gaussian, noise free, single source, single 
    vehicle scenario}
\newacronym{DNFSSSV}{D-NF-SS-SV}{Gaussian dispersion, noise free, single source, 
    single vehicle scenario}
\newacronym{DNFMSSV}{D-NF-MS-SV}{Gaussian dispersion, noise free, multiple 
    source, single vehicle scenario}
\newacronym{DSNSSSV}{D-SN-SS-SV}{Gaussian dispersion, sensor noise, single 
    source, single vehicle scenario}
\newacronym{DSNMSSV}{D-SN-MS-SV}{Gaussian dispersion, sensor noise, multiple 
    source, single vehicle scenario}
\newacronym{DSNMSMV}{D-SN-MS-MV}{Gaussian dispersion, sensor noise, multiple 
    source, multiple vehicle scenario}
\newacronym{EU}{EU}{European Union}
\newacronym{NED}{NED}{north, east, down}
\newacronym{SD}{SD}{standard deviation}
\newacronym{norm}{norm.}{normalized}
\glsadd{GNFSSSV}
\glsadd{DNFSSSV}
\glsadd{DNFMSSV}
\glsadd{DSNSSSV}
\glsadd{DSNMSSV}
\glsadd{DSNMSMV}
\glsadd{EU}
\glsadd{NED}
\glsadd{SD}
\glsadd{norm}

\addbibresource{references.bib}

\title{Gaussian Processes for Plume Distribution Estimation with UAVs}
\author{Jan Gosmann}
\makeatletter
\hypersetup{
  pdftitle={\@title},
  pdfauthor={\@author}
}
\makeatother

\newcommand{\addsym}[3]{\newglossaryentry{#1}{sort={#1},name={\ensuremath{#2}},description={#3}}\glsadd{#1}}
\newcommand{\addspecialsym}[4]{\newglossaryentry{#1}{sort={#2},name={\ensuremath{#3}},description={#4}}\glsadd{#1}}
\addspecialsym{sim}{~}{\sim}{distributed according to}
\addspecialsym{blockmat}{[]}{[\ ]}{matrix in block notation}
\addsym{0}{\vc{0}}{null vector $(0, \dots, 0)\Tr$}
\addsym{A}{\norm{\mat{A}}}{norm of $\mat A$}
\addsym{Aij}{(\mat{A})_{ij}}{element of $\mat A$ at row $i$ and column $j$}
\addsym{AT}{\mat{A}\Tr}{the transpose of $\mat A$}
\addsym{BV}{\mathcal{BV}}{set of basis vectors}
\addsym{cov}{\cov(\vc{x})}{covariance (matrix) of elements of $\vc{x}$}
\addsym{Ct}{-\mat{C_t}}{with $t$ samples sparsely approximated covariance 
    matrix}
\addsym{cx}{c (\vc{x})}{true concentration distribution}
\addsym{D}{\mathcal{D}}{set of combined training inputs and targets $(\vc x_i, 
    y_i)$}
\addsym{dxx}{d (\vc{x}, \vc{x}')}{Euclidean distance (unitless)}
    \addsym{det}{\det{}\mat{A}}{the determinant of $\mat A$}
\addsym{diagx}{
    \diag\vc{x}}{diagonal matrix with the elements of the vector $\vc x$}
\addsym{diagA}{
    \diag\mat{A}}{vector with the diagonal elements of the matrix $\mat A$}
\addsym{E}{E}{error measure}
\addsym{Ehat}{\hat{E}}{estimate of an error}
\addsym{F}{F}{normalized error}
\addsym{Ktilde}{\tilde{\mat{K}}}{covariance matrix including the noise variance 
    $\sigma\ped{n}^2$ on the diagonal}
\addsym{k}{k (\vc{x}, \vc{x}')}{some kernel, usually used as covariance 
    function}
\addsym{K}{K (X, X')}{matrix of pairwise covariances of $\vc x \in X$ and $\vc 
    x' \in X'$}
\addsym{Kvz}{K_{\nu} (z)}{modified Bessel function}
\addsym{L}{\mat{L}}{Cholesky factor (lower, triangular matrix)}
\addsym{ln}{\ln}{natural logarithm}
\addsym{N}{N (x; \mu, \sigma^2)}{
    Gaussian probability density at $x$ with mean~$\mu$ and variance~$\sigma^2$}
\addsym{Ncal}{\mathcal{N} (\vc{m}, \mat{\Sigma})}{
    multivariate normal distribution with mean $\vc m$ and covariance matrix 
    $\mat \Sigma$}
\addsym{mslim}{\mslim}{mean square limit}
\addsym{p}{p (x)}{probability or probability density of $x$}
\addsym{R}{R}{QRSim reward}
\addsym{sy}{s (\vc{y})}{Scaling factor in an acquisition function}
\addsym{ux}{u (\vc{x})}{utility function}
\addsym{V}{V}{the simulated volume}
\addsym{x}{\vc{x}}{some location or input data}
\addsym{X}{X}{set of training inputs}
\addsym{Xasterisk}{X_*}{set of test inputs}
\addsym{xabs}{\abs{\vc{x}}}{Euclidean ($L^2$) norm of $\vc x$}
\addspecialsym{Phi}{Φ}{\Phi(x; \mu, \sigma^2)}{
    Gaussian cummulative distribution function at $x$ with mean~$\mu$ and 
    variance~$\sigma^2$}
\addsym{xmean}{\bar{x}}{mean of the random variable $x$}
\addsym{xi}{x_i}{$i$-th component of the vector $\vc x$}
\addsym{xy}{x | y}{conditional random variable $x$ given $y$}
\addsym{y}{\vc{y}}{vector of training targets}
\addspecialsym{gamma}{γ}{\gamma}{distance penalty weighting}
\addspecialsym{Gamma}{Γ}{\Gamma(\nu)}{Gamma function}
\addspecialsym{kappa}{κ}{\kappa}{variance weighting}
\addspecialsym{mu}{μ}{\mu(\vc{x})}{mean predicted by a Gaussian process 
    (unitless)}
\addspecialsym{sigma}{σ}{\sigma^2(\vc{x})}{variance predicted by a Gaussian 
    process (unitless)}
\addspecialsym{sigmas}{σk}{\sigma\ped{k}^2}{kernel process variance}
\addspecialsym{sigman}{σn}{\sigma\ped{n}^2}{noise variance of the Gaussian 
    process}
\addspecialsym{sigmasn}{σsn}{\sigma\ped{sn}^2}{sensor noise variance}
\addspecialsym{sigmat}{σt}{\sigma\ped{t}^2}{novelty of input $\vc x_t$}
\addspecialsym{rho}{ρ}{\rho}{weighting of distance to other UAVs}

\begin{document}
\frontmatter
\begin{titlepage}
    \begin{center}
        \vspace*{2.5cm}
        \textsf{\Large Master Thesis}\\
        \vspace{1cm}
        \begin{spacing}{2.5}
            \textsf{\textbf{\huge Gaussian Processes for Plume Distribution 
                    Estimation with UAVs}}
        \end{spacing}
        \vspace{1cm}
        \Large
        \begin{spacing}{1.5}
            Jan Gosmann\\
            \today
        \end{spacing}
        \vspace{\fill}
        \includegraphics[width=6cm]{plots/vis_multi_dispersion}\\
        \vspace{\fill}
        Supervisors:\\
        Prof.~Dr.~Manfred~Opper\\
        Dr.~Andreas~Ruttor
    \end{center}
\end{titlepage}
\cleardoublepage{}

\thispagestyle{empty}
\begin{german}
    \vspace*{\fill}
    \noindent Hiemit erkläre ich, dass ich die vorliegende Arbeit selbstständig 
    und eigenhändig sowie ohne unerlaubte fremde Hilfe und ausschließlich unter 
    Verwendung der aufgeführten Quellen und Hilfsmittel angefertigt habe.

    \vspace{\intextsep}
    \noindent Berlin, den \today

    \vspace{\intextsep}
    \vspace{\intextsep}
    \noindent Jan Gosmann
    \vspace*{\fill}
    \vspace*{\fill}
\end{german}

\cleardoublepage\thispagestyle{empty}
\vspace*{\fill}
\section*{\abstractname}
Recent scientific work explored the possibility to use mobile robots for 
environmental monitoring. This includes for example the estimation of ozone 
concentrations or locating the source of a pollutant plume. So far the modeling 
of the complete spatial distribution of a plume (which has different spatial 
characteristics compared to the ozone concentrations) has not been done.  In 
this work existing methods of Bayesian optimization, namely global optimization 
(GO) and the distance based upper confidence bound (DUCB), were be evaluated on 
this task.  Also, a new method~-- plume distance based upper confidence bound 
(PDUCB)~-- and an extension to multiple robots is proposed.  All methods were 
tested in simulations using the QRSim quadrotor simulator. The existing methods 
were not able to solve the task satisfyingly, whereas the PDUCB method was able 
to approximate plume distributions with noisy measurements reasonably well.

\begin{german}
\section*{\abstractname}
Neuere wissenschaftliche Arbeiten haben die Möglichkeit untersucht mobile 
Roboter zur Umweltüberwachung einzusetzen. Dies beinhaltet die Vermessung von 
Ozonkonzentrationen und die Lokalisierung der Quelle einer Gasfahne. Bisher 
wurde jedoch nicht versucht die komplette räumliche Verteilung einer Gasfahne 
(welche andere räumliche Charisitika im Vergleich zu Ozonkonzentrationen hat) zu 
bestimmen.  In dieser Arbeit wurde die Anwendbarkeit globaler Optimierung (GO) 
und der distanzbasierten oberen Konfidenzgrenze (DUCB), bestehende Methoden 
Bayes'scher Optimierung, für diese Aufgabe beurteilt.  Weiterhin wird eine neue 
Methode~-- die distanzbasierte obere Konfidenzgrenze für Gasfahnen (PDUCB)~-- 
sowie eine Erweiterung für mehrere Roboter vorgeschlagen. Alle Methoden wurden 
in Simulationen mit dem QRSim Quadrotorsimulator getestet. Die bestehenden 
Methoden waren nicht in der Lage die Aufgabe zufriedenstellend zu lösen, 
wohingegen die PDUCB-Methode in der Lage war die Verteilungen der Gasausbreitung 
angemessen gut mit verrauschten Messungen anzunähern.
\end{german}
\vspace*{\fill}

\tableofcontents

\setglossarypreamble{Variables are typeset in italics, whereas constants are 
    upright.  Vectors and matrices use a bold font. Additionally, matrices use 
    uppercase letters.}

\printglossary[title={Symbols and Notation}]

\printglossary[type=\acronymtype]

\mainmatter{}
\chapter{Introduction}
Environmental monitoring is used to ensure water and air pollution levels are in
compliance with governmental regulations \parencite[i.\,e.][]{Anonymous:1996ui}, 
to monitor ozone concentrations and climate change, or for surveillance of 
industrial facilities for leakages of pollutants to just name a few 
applications.

In many of these scenarios it is feasible to have a static sensor network.  
Therefore, it is not surprising that research on optimal sensor placement at 
fixed locations exists \parencite[e.\,g.][]{Osborne:2008hi, Guestrin:2005cq, 
    Wang:kz}.  However, better results might be obtainable using mobile robots 
which can move to interesting areas and acquire more precise data there.  
Moreover, in some scenarios like disaster response, where timely information is 
needed, it might not be possible to first deploy an extensive sensor network. In 
this case mobile robots allow here to quickly identify the interesting regions.  
The problem of autonomously choosing the best locations for data acquisition is 
known as \newterm{active learning}. In the setting of environmental surveillance 
\textcite{Marchant:2012wb} also used the term \newtermAbbrev{intelligent 
    environmental monitoring}{IEM}.

In this work, I focus on a scenario proposed as part of the CompLACS project in 
\textcite{denardi2013rn}: One or more sources emit a gaseous substance or 
aerosol which is dispersed by a constant wind. The resulting plume distribution 
has to be estimated with autonomously controlled \pabbrev{unmanned aerial 
    vehicle}{UAV}. The rather steep concentration gradients and small spatial 
extent orthogonal to the main dispersion axis add to the difficulty of this 
problem.  With a few UAVs it is not possible to cover the whole volume of 
investigation using a regular pattern densely in a timely manner.  It is 
necessary to focus on measurements in specific areas.  Furthermore, measurement 
noise has to be taken into consideration.

In previous works swarms of robots have been used to localize the source of 
a plume \parencite{Jatmiko:2007df, Zarzhitsky:2005tz}. These approaches, 
however, do not allow the usage of only one robot and do not provide one with an 
estimation of the overall plume distribution. Such an estimation might be 
important for various reasons such as determining areas in which a threshold is 
exceeded or a contamination occurred. As \textcite{Reggente:2009ti} noted, 
though one could try to model the actual fluid dynamics to obtain this 
information, such computational fluid dynamics models become intractable for 
real world applications with inaccurate data.  Instead they propose to build 
a statistical model with the gas concentration measurements as random variables.

A widely used statistical model for spatial or spatio-temporal data are 
\newterm{Gaussian processes}\footnote{In geospatial statistics the modeling 
    with Gaussian processes is also known as \newterm{kriging}.}. In several 
works \parencite[e.\,g.][]{Stachniss:2008vz, Marchant:2012wb} the modeled data 
were actually gas concentrations. Also, there exists some prior work on actively 
selecting the sampling locations. \Textcite{Stranders:2008wl} do this for 
discrete locations; \textcite{Singh:2010wt} and \textcite{Marchant:2012wb} for 
continuous locations. However, none of these approaches is optimal for plume 
dispersions. This work will porpose and evaluate the improved PDUCB method for 
the given task.

The thesis is organized as follows. First, a description of the plume modeling 
scenarios will be given establishing the background of this work.  
Chapter~\ref{sec:gp} will give a general introduction into Gaussian Processes 
and discusses some topics specifically related to the modeling of plume 
distributions including online updates and active learning. Following in 
Chapter~\ref{sec:error} a number of error measures will be introduced needed to 
evaluate different approaches. Some further details on how the algorithms were 
implemented are given in Chapter~\ref{sec:tech}. The results of a number of 
simulation experiments are presented in Chapter~\ref{sec:exp}, before providing 
a short outlook on modeling time-varying plume distributions in 
Chapter~\ref{sec:timevarying}. Finally, a conclusion will be given.

\chapter{The QRSim Plume Modelling Scenarios}
The general plume modeling scenario as tackeled in this thesis is part of the 
QRSim quadrotors simulator \parencite{denardi2013rn}. Several task variations 
were proposed from which I chose a selection and to which I added some 
modifications of my own.  The task scenarios can be classified along four 
dimensions: type of dispersion (G, D), presence of sensor noise (NF, SN), single 
or multiple pollutant sources (SS, MS), single or multiple vehicles (SV, MV).

As long as not otherwise noted location vectors are in the NED (north, east, 
down) reference frame.  Hence, the height of a location $\vc x = (x_1, x_2, 
x_3)\Tr$ is given by $-x_3$.

The most simple scenario is a Gaussian (G) plume without wind as shown in 
Figure~\ref{fig:SS-G}.  The pollutant is emitted at a constant rate resulting in 
a three-dimensional (potentially non-isotropic) Gaussian plume distribution.  
Given a source location $\vc s$, covariance matrix $\mat \varSigma$, and 
emission rate $Q$ in \si{\gram\per\second} the concentration $c(\vc x)$ at 
location $\vc x$ is given as
\begin{equation}
    c(\vc x) = Q \cdot \si{\second\per\meter^3} \cdot \exp\!\del{-\frac{1}{2} 
        (\vc x - \vc s)\Tr \mat \varSigma^{-1} (\vc x - \vc s)}.
\end{equation}

\begin{figure}
    \centering
    \subfigure[Single source 
    Gaussian]{\includegraphics[width=4.5cm]{plots/vis_gaussian}\label{fig:SS-G}}%
    \subfigure[Single source 
    dispersion]{\includegraphics[width=4.5cm]{plots/vis_dispersion}\label{fig:SS-D}}%
    \subfigure[Multiple source 
    dispersion]{\includegraphics[width=4.5cm]{plots/vis_multi_dispersion}\label{fig:MS-D}}
    \caption[Visualizations of plume dispersions]{Visualization of different 
        plume dispersions. The rear boundaries of the volume show 
        two-dimensional projections of concentration maxima in the respective 
        directions. Axes scale is in meters.}
\end{figure}

A Gaussian dispersion (D) as shown in Figure~\ref{fig:SS-D} is obtained when 
considering a constant wind parallel to the ground with velocity $u$ measured 
\SI{6}{\meter} above the ground. The plume will be dispersed and form 
a cone-like distribution along the wind direction.  Making a few more 
assumptions (constant $Q$, steady-state, isotropic diffusion, no ground 
penetration, and neglegible variation in topography) the analytic expression
\begin{multline}\label{eqn:plumedisp}
    c(\vc x') = \frac{Q}{2\uppi ua\del{x'_1 - s'_1}^b} 
    \exp\!\del{-\frac{\del{x'_2 - s'_2}^2}{2a\del{x'_1 - s'_1}^b}} \\ 
    \sbr{\exp\!\del{-\frac{\del{x'_3 - s'_3}^2}{2a\del{x'_1 - s'_1}^b}} 
        + \exp\!\del{-\frac{\del{x'_3 + s'_3}^2}{2a\del{x'_1 - s'_1}^b}}}
\end{multline}
can be derived for the concentration \parencite{Stockie:2011fd}. Note that the 
coordinates $\vc x'$ and $\vc s'$ are expressed in the wind frame of reference.  
In the scenario the wind speed is set to $u = \SI{3}{\meter\per\second}$ and the 
diffusion parameters to $a = \SI{0.33}{\meter\tothe{2 - \mathit{b}}}$ and $b 
= 0.86$.  The emission rate $Q$ is randomly chosen from a uniform distribution 
over the interval \SIrange{0.1}{2.5}{\gram\per\second}.

Sensor noise (SN) of the plume sensor is assumed to be additive and distributed 
according to $\mathcal{N}(0, \sigma\ped{sn}^2)$. In the noise free (NF) 
scenarios no noise was added to the measurements. The standard deviation 
$\sigma\ped{sn}$ was set to $\SI{e-5}{\gram\per\meter\cubed}$ in the scenarios 
including noise.  The QRSim default scenarios set it to 
$\SI{e-2}{\gram\per\meter\cubed}$, but given the low default plume concentration 
this would require roughly an averaging of 385 samples from one single location 
to reduce the magnitude of the noise below the magnitude of the concentration 
values (see Appendix~\ref{sec:decnoise}).  Hence, the default scenario is not 
solvable in a feasible amount of simulation time.

The overall concentration $c(\vc x)$ for $n$ sources like in 
Figure~\ref{fig:MS-D} is obtained by summing the individual contributions 
$c_i(\vc x)$ for each source:
\begin{equation}
    c(\vc x) = \sum_{i = 1}^n c_i(\vc x)
\end{equation}
In the scenarios with multiple sources (MS) $n$ is chosen uniformly out of the 
range from \numrange{1}{5}. With a single source (SS) $n = 1$ is fixed. In both 
cases the source locations will be randomly chosen from a uniform distribution 
over the simulated volume.

The starting locations of the UAVs are also chosen randomly and uniformly in the 
simulated area, but the height is initially set to $x_3 = \SI{-10}{\meter}$.  
Either a single UAV (SV) or three UAVs (MV) were used.

This gives a number of possible scenarios from which I focussed on the following 
in this work:
\begin{itemize}
    \item In Chapter~\ref{sec:cmputility} I discuss the noise free, single 
        vehicles scenarios G-NF-SS-SV, D-NF-SS-SV, and D-NF-MS-SV\@. The first 
        is equivalent to the scenario 3A in \textcite{denardi2013rn}.
    \item In Chapter~\ref{sec:noisy} I consider the dispersion scenarios with 
        noise D-SN-SS-SV and D-SN-MS-SV\@. Except for the amount of noise these 
        correspond to scenarios 3B and 3C in \textcite{denardi2013rn}.
    \item Finally, I will take a look at the usage of multiple vehicles with the 
        scenario D-SN-MS-MV corresponding to scenario 3D in 
        \textcite{denardi2013rn}.
\end{itemize}

\chapter{Gaussian Processes}\label{sec:gp}
A vast number of regression methods have been proposed in the machine learning 
literature. In this work I use Gaussian Processes as these have been 
successfully used in a number of studies related to spatial and environmental 
monitoring including the modelling of gas distributions 
\parencite[e.\,g.][]{Stranders:2008wl, Marchant:2012wb, Stachniss:2008vz}.  
Gaussian Processes exhibit a number of desirable features. They are 
non-parametric, non-linear and therefore do not require any assumptions about 
the underlying functions or limitations of the search space.  Also, they provide 
an estimate of the predictive uncertainties which can be used for a natural 
exploration-exploitation trade-off.

In the remainder of this chapter I will discuss the essentials of Gaussian 
Process regression. A more thorough introduction can be found in 
\textcite{Rasmussen:2006vz}.

Let $X = \{\vc x_i | i = 1, \dots, n\}$ be a set of training inputs and $\vc 
y = (y_1, \dots, y_n)\Tr$ a vector of targets. The individual targets are 
assumed to follow $y_i = f(\vc x_i) + \eta$ with additive noise $\eta \sim 
\mathcal{N}(0, \sigma\ped{n}^2)$. The complete set of training data will be 
denoted with $\mathcal{D} = \{(\vc x_i, y_i) | i = 1, \dots, n\}$. We want to 
learn the function $f(\vc x)$ from this training data.

A \newterm{Gaussian Process}
\begin{equation}
    f(\vc x) \sim \mathcal{GP}(m(\vc x), k(\vc x, \vc x'))
\end{equation}
imposes a multivariate Gaussian distribution on the space of functions $f(\vc 
x)$. It is completely specified by the mean function $m(\vc x)$ and covariance 
function $k(\vc x, \vc x')$. Usually, though not necessarily, the mean function 
is taken to be zero. In most scenarios the choice of the covariance function is 
much more interesting as it controls features like the smoothness of the 
predicted underlying function. I will discuss this topic regarding the modelling 
problem on hand in Section~\ref{sec:covfn}.

We can now formulate the joint Gaussian prior distribution of the observed 
training targets and the predicted values $\vc f_*$ at unseen locations $X_*$ 
(assuming $m(\vc x) = 0$):
\begin{equation}
    \left[ \begin{array}{c}\vc y \\ \vc f_* \end{array} \right]
    \sim\mathcal{N}\!\del{\vc{0}, \sbr{\begin{array}{cc} K(X, X) 
                + \sigma\ped{n}^2 \mat I & K(X, X_*) \\ K(X_*, X) & K(X_*, X_*) 
            \end{array}}}
\end{equation}
Here $K(X, X')$ are matrices with the elements $(i, j)$ being the covariances 
$k(\vc x_i, \vc x'_j)$ evaluated for all pairs $\vc x_i \in X$ and $\vc x'_j \in 
X'$. In the following I will use $\tilde{\mat K} = K(X, X) + \sigma\ped{n}^2 
\mat I$ as a shorter notation. By conditioning on the observations one obtains 
the predictive distribution for $\vc f_*$ as
\begin{align}
    \vc f_* | X, \vc y, X_* &\sim \mathcal{N}(\bar{\vc f_*}, \cov(\vc 
    f_*))\text{, with}\\
    \bar{\vc f_*} &= \mu(X_*) = K(X_*, X)\tilde{\mat K}^{-1} \vc y\text{,} 
    \label{eqn:meanpred} \\
    \cov(\vc f_*) &= K(X_*, X_*) - K(X_*, X)\tilde{\mat K}^{-1}K(X, X_*) 
    \label{eqn:covpred} \\
    \sigma^2(X_*) &= \diag(\cov(\vc f_*)) \text{.}
\end{align}
See Figure~\ref{fig:ex-gp-main} for a visualization of an example Gaussian 
process.

\begin{figure}
    \centering
    \subfigure[]{\includegraphics{plots/gp}\label{fig:ex-gp}}%
    \subfigure[]{\includegraphics{plots/gp_sample}\label{fig:ex-gp-sample}}
    \caption[Gaussian process example]{Example of a one-dimensional Gaussian 
        process (using the squared exponential covariance function with 
        a length scale of 1, $\sigma\ped{n}^2 = 0$) conditioned on five training 
        points: \subref{fig:ex-gp} shows the mean $\bar{\vc f_*}$ and predictive 
        standard deviation $\sigma(X_*)$; \subref{fig:ex-gp-sample} shows three 
        functions sampled from the process.}\label{fig:ex-gp-main}
\end{figure}

Even though, $K$ is a symmetric, positive-definite matrix it can be 
ill-conditioned\footnote{The condition $\kappa(\mat A)$ of a matrix $\mat A$ is 
    defined as $\kappa(\mat A) = \|\mat A\| \|\mat A^{-1}\|$. Using the 
    $L^2$-norm this corresponds to $\kappa(\mat A) = \lambda_1/\lambda_n$,
    the ratio of the largest eigenvalue $\lambda_1$ and the smallest one
    $\lambda_n$. If the condition number $\kappa(\mat A)$ is too large, the 
    matrix is near-singular and ill-conditioned.} and lead to numerical 
instabilities. This happens especially for close-by input data points as they 
occur in a sequential scenario like the plume modelling here.

There are two commonly implemented approaches to counteract the problem of 
ill-conditioning \parencite[cp.]{Sacks:1989cv, Neal:1997tj, Booker:1999wz, 
    Gramacy:2008es}. Firstly, instead of using a general matrix inversion 
algorithm one can utilize the symmetry and positive-definiteness of 
$\tilde{\mat{K}}$ by doing a Cholesky decomposition. This yields a lower, 
triangular matrix $\mat L$ satisfying $\tilde{\mat K} = \mat L\mat L\Tr$. The 
inverse can the be calculated as $\tilde{\mat K}^{-1} = (\mat L^{-1})\Tr \mat 
L^{-1}$.  Secondly, a well conditioned $\tilde{\mat K}$ can be ensured by adding 
a nugget $g > 0$ (also known as jitter) to the diagonal of the covariance 
matrix. This will increase all eigenvalues by the same value and thus improve 
the condition.  The addition of a nugget can also be seen as increasing the 
noise variance $\sigma\ped{n}^2$ and thus allowing the Gaussian Process to match 
the target less precisely and to become smoother.

\section{Online Updates}\label{sec:onlineup}
A naive implementation requires a $O\!\del[1]{\del[0]{n + N}^3}$ matrix 
inversion whenever new data points are added to the Gaussian Process with $N$ 
being the total number of data points collected so far and $n$ being the number 
of new data points.  However, it is possible to do online updates where only 
a $n \times n$ matrix has to be inverted.  This reduces the complexity of the 
matrix inversion to $O(n^3)$ and the overall complexity including the necessary 
matrix multiplications to $O(n \max\{n^2, N^2\})$.

Let us denote the set of inputs already trained on with $X$ and the set of 
inputs to add as $X'$. The block covariance matrix after adding these new inputs 
will be
\begin{equation} \label{eqn:tilde_K_prime}
    \tilde{\mat K}' = \left[ \begin{array}{cc}
            \tilde{\mat K} & K(X, X') \\ K(X', X) & K(X', X') 
            + \sigma\ped{n}^2\mat I
        \end{array}
    \right]\text{.}
\end{equation}
The Cholesky factorization can also be written with block matrices
\begin{equation}
    \tilde{\mat K}' = \mat L' {\mat L'}\Tr = \left[
        \begin{array}{cc}
            \mat L & \mat 0 \\ \mat A & \mat B
        \end{array}
    \right] \left[
        \begin{array}{cc}
            \mat L\Tr & \mat A\Tr \\ \mat 0 & \mat B\Tr
        \end{array}
    \right] = \left[
        \begin{array}{cc}
            \mat L \mat L\Tr & \mat L \mat A\Tr \\ \mat A \mat L\Tr & \mat A \mat 
            A\Tr + \mat B \mat B\Tr
        \end{array}
    \right]
\end{equation}
and comparison with equation~(\ref{eqn:tilde_K_prime}) gives the following 
relations:
\begin{align}
    \mat A &= K(X', X) \del{\mat L\Tr}^{-1} \\
    \mat B \mat B\Tr &= K(X', X') + \sigma\ped{n}^2\mat{\mathrm{I}} - K(X', 
    X)\tilde{\mat{K}}^{-1}K(X, X')
\end{align}
As $\mat B \mat B\Tr$ is symmetric, positive-definite it is possible to obtain 
$\mat B$ also by a Cholesky decomposition.
With the inverse of a block matrix \parencite[45]{Petersen:2008wc} we obtain the 
following relation for the inverse of the updated Cholesky factor:
\begin{equation}
    \mat L'^{-1} = \left[
        \begin{array}{cc}
            \mat L^{-1} & \mat 0 \\ -\mat B^{-1} K(X', X)\tilde{\mat K}^{-1} 
            & \mat B^{-1}
        \end{array}
    \right] \label{eqn:invChol}
\end{equation}

\section{Sparse Approximations}
Despite online updates there is an quadratic increase in the complexity for 
adding new training data points to a Gaussian process. This efficiency problem 
can be alleviated by using a sparse approximation. A number of different methods 
has been proposed \parencites[chapter~8 
in][]{Rasmussen:2006vz}{QuinoneroCandela:2005wp}. Unfortunately, these methods 
usually assume that all training data is already accessible which is not the 
case in an online scenario. An exception to this is the online approximation by 
\textcite{Csato:2002fp}.

They replace the inverse covariance matrix $\tilde{\mat K}_t^{-1}$ in 
Equation~\ref{eqn:covpred} after $t$ updates by $-\mat C_t$. Normal (full) 
online updates are performed as discussed in the previous section by updating 
$-\mat C_t$ respectively the Cholesky factor $\mat L_t^{-1}$.\footnote{The 
    original paper formulates the update a bit different. The equivalence for 
    full updates is shown in Appendix~\ref{sec:sparse-gp-apdx}.} If, however, 
the ``novelty''
\begin{equation}
    \tilde{\sigma}^2_{t+1} = k(\vc x_*, \vc x_*) + \sigma\ped{n}^2 
    \mat{\mathrm{I}} - K\!\del{\{\vc x_*\}, X_t} \tilde{\mat K}_t^{-1} 
    K\!\del{X_t, \{\vc x_*\}}
\end{equation}
of a new input $\vc x_*$ is below a threshold $\epsilon\ped{tol}$, a reduced 
update will be performed. The reduced update leaves the size of $\mat C_t$ 
unchanged.

All inputs used for a full update are called basis vectors and constitute the 
set~$\mathcal{BV}$. The size of this set can be limited by deleting the basis 
vector with the smallest error whenever the limit is exceeded. The error 
associated with the $i$-th basis vector is given by
\begin{equation}
    e_i = \frac{\abs[1]{\del[1]{\tilde{\mat K}_{t+1}^{-1} 
                \vc{y}}_i}}{\del[1]{\tilde{\mat{K}}_{t+1}^{-1}}_{ii}} \text{.}
\end{equation}

The basis vector deletion requires downdating the covariance matrix 
$\tilde{\mat{K}}_{t+1}$ and performing a reduced updated with the deleted basis 
vector. Unfortunately, as shown in Appendix~\ref{sec:sparse-gp-apdx} the 
calculation of $-\mat C_t$ is based on the Cholesky factors and downdating based 
on Cholesky factorization or a matrix obtained from the factorization is known 
to be numerical unstable \parencite{Bjorck:1994dz}, especially when the 
correlation of the training inputs is high. Using other factorizations like the 
QR factorization a more stable downdating algorithm can be obtained.  However, 
this comes at the cost of the initial factorization being numerical more 
unstable.  Also, it would be possible to directly downdate 
$\tilde{\mat{K}}_{t+1}$ and recalculate the Cholesky factorization. Yet, this 
leads to cubic instead of quadratic complexity for downdating.

It turned out that these issues with numerical stability do not make this sparse 
online approximation a viable option for plume distribution estimation. The 
collected samples are highly correlated and make downdating the Cholesky factor 
too unstable, while using a more stable method impairs performance below the 
non-sparse Gaussian process level. By resigning from deleting basis vector and 
only using reduced updates based on the novelty $\tilde{\sigma}^2_{t+1}$ one 
does not gain much. It only considers the spatial relation of inputs, but not 
the actual error in prediction at those locations. Hence, one would either still 
use almost all data for full updates or use not enough full updates in areas of 
high concentration where a close sampling is necessary. Luckily, the performance 
of the non-sparse Gaussian processes was sufficient as at most only a few 
thousand training inputs were used.

\section{Covariance Functions}\label{sec:covfn}
The choice of the covariance function determines the assumptions about the 
functions learned with a Gaussian process. Hence, it is quite important. In this 
chapter, I will discuss some widely used covariance functions and considerations 
to take into account. A more thorough discussion including further covariance 
functions is to be found in \textcite[Chapter 4]{Rasmussen:2006vz} on which this 
section is based.

A valid covariance function $k(\vc x, \vc x')$ has to be a kernel 
\parencite{Cressie:1993uu} satisfying semi-positive definiteness
\begin{equation}
    \int f(\vc x) k(\vc x, \vc x') f(\vc x') \dif \vc x \dif \vc x' \geq 
    0 \text{.}
\end{equation}
This ensures that the kernel's Gram matrix for a set of inputs $\cbr{x_i 
    | i = 1, \dots, n}$ with entries $\mat K_{ij} = k(\vc x_i, \vc x_j)$ is also 
semi-positive definite and therefore a valid, invertible covariance matrix.

The \newterm{smoothness} of the Gaussian process is also determined by the 
covariance function.  This is formalized in the notion of how many times it is 
\newtermAbbrev{mean square}{MS}\newterm{ differentiable}.  A process $f(\vc x)$ 
is differentiable if the mean square limit denoted by $\mslim$ in the mean 
square derivative exists. The MS derivative is given by
\begin{equation}
    \dpd{f(\vc x)}{x_i} = \mslim_{h \rightarrow 0} \frac{f(\vc x + h\vcc e_i) 
    - f(\vc x)}{h}
\end{equation}
for the $i$-th direction with the unit vector $\vcc e_i$.

%defined in \textcite[81]{Rasmussen:2006vz} in the following way: ``Let $x_1, 
%x_2, \dots$ be a sequence of points and $x_*$ be a fixed point in $\mathbb{R}^D$ 
%such that $\abs{x_k - x_*} \rightarrow 0$ as $k \rightarrow \infty$. Then 
%a process $f(x)$ is continuous in mean square at $x_*$ if 
%$\mathbb{E}[\abs[0]{f(x_k) - f(x_*)}^2] \rightarrow 0$ as $k \rightarrow 
%\infty$. If this holds for all $x_* \in A$ where $A$ is a subset of 
%$\mathbb{R}^D$ then $f(x)$ is said to be continuous in mean square (MS) over 
%$A$.''

\subsection{Stationary Covariance Functions}
A kernel which is only a function of $\vc x - \vc x'$ is called 
\newterm{stationary} and is invariant to translations. Furthermore, it is 
\newterm{isotropic} if it is a \abbrev{radial basis function}{RBF} $k(r)$ with 
    $r = \abs{\vc x - \vc x'}$.  An isotropic kernel is invariant to all rigid 
    motions.

For stationary kernels the smoothness properties of the resulting Gaussian 
process can be easily obtained: It is $k$-times MS differentiable if at $\vc 
x = \vc 0$ the $2k$-th order partial derivatives $\partial^{2k} k(\vc x) 
/ \partial x_{i_1}^2 \dots \partial x_{i_k}^2$ exist and are finite. Thus, the 
process smoothness is essentially determined by the kernel properties around 
$\vc 0$.

A common default choice is the \newtermAbbrev{squared exponential}{SE} kernel 
    defined as
\begin{equation}
    k\ped{SE}(r) = \sigma\ped{k}^2 \exp\del{-\frac{r^2}{-2\ell^2}}
\end{equation}
with the desired process variance $\sigma\ped{k}^2$ and length scale $\ell$. It 
produces infinitely MS differentiable Gaussian processes. This can, actually, be 
too smooth in many applications.

The \newterm{Mat\'ern class} of covariance functions allows to control the 
smoothness with a parameter $\nu$. Using the modified Bessel function $K_{\nu}$ 
it is given by
\begin{equation}
    k_{\nu}(r) = \sigma\ped{k}^2 \frac{2^{1-\nu}}{\Gamma(\nu)} 
    \del{\frac{r\sqrt{2\nu}}{\ell}}^{\nu} K_{\nu} 
    \del{\frac{r\sqrt{2\nu}}{\ell}}
\end{equation}
The resulting Gaussian process will be $k$ times MS differentiable for $k 
< \nu$. The parameter $\ell$ denotes again the characteristic length scale.

Typically, only the kernels with $2\nu \in \cbr{1, 2, 3}$ are used.  Usually it 
is not possible to tell which kernel leads to a better fit for larger $\nu$ from 
the noisy data.  Half-integer values are used as the kernel function will become 
quite simple:
\begin{align}
    k_{5/2}(r) &= \sigma\ped{k}^2 \del{1 + \frac{r\sqrt{5}}{\ell} 
        + \frac{5r^2}{3\ell^2}} \exp\del{-\frac{r\sqrt{5}}{\ell}} \\
    k_{3/2}(r) &= \sigma\ped{k}^2 \del{1 + \frac{r\sqrt{3}}{\ell}} 
    \exp\del{-\frac{r\sqrt{3}}{\ell}} \\
    k_{1/2}(r) &= k_{\exp}(r) = \sigma\ped{k}^2 \exp\del{-\frac{r}{\ell}}
\end{align}
From these kernel functions $k_{\nu=1/2}(r)$ is also known as the 
\newterm{exponential kernel}. Furthermore, note that for $\nu \rightarrow 
\infty$ the squared exponential kernel is recovered. A plot of the different 
covariance functions can be found in Figure~\ref{fig:kernels}.

\begin{figure}
    \centering
    \includegraphics{plots/kernels}
    \caption[Covariance functions]{Plot of stationary covariance functions with 
        $\sigma_k^2 = 1, \ell = 1$.}\label{fig:kernels}
\end{figure}

\subsection{Non-stationary Covariance Functions}
Many phenomena, including the concentrations of gas plumes, are not stationary.  
Already a Gaussian density function exhibits different optimal length scales 
(see Figure~\ref{fig:gp-length scale}).  With a long length scale the predicted 
mean can considerably deviate from the target function as shown for positive $x$ 
in the figure. With a short length scale a relatively good fit is obtained.  
However, the predictive variance along the tail towards negative $x$ is 
overestimated as the actual rate of change in this area is quite low.

\begin{figure}
    \centering
    \includegraphics{plots/gp-lengthscale}
    \caption[Length-scale influence]{Influence of the kernel length scale on the 
        Gaussian process. In both plots the Mat\'ern kernel with $\nu = 3/2$ was 
        used. On the left a short length scale of $\ell = 1$ was used, whereas 
        a longer length scale of $\ell = 10$ was used on the right.
    }\label{fig:gp-length scale}
\end{figure}

Non-stationary covariance functions can alleviate this problem. Moreover, they 
allow to model discontinuities at specific places. However, the usage of 
non-stationary covariance functions for the given plume modelling problem is far 
from straightforward and might need more prior knowledge than one is willing to 
assume (in simulations) or effectively has.  One would probably have to use 
different kernels depending on the scenario (wind/no wind, number of sources) 
and these would have to be parameterized with the source locations. Otherwise 
the non-stationarity of the kernel could not relate to the actual 
non-stationarity of the plume.

Methods for selecting such parameters will be discussed in the next section.  
Unfortunately, the cost of these methods grows with the number of parameters 
which for non-stationary kernels will be larger. Matters are complicated even 
more as it is usually desirable to have a differentiable kernel to be able to 
use gradient-based optimizers. Moreover, In an active learning scenario there is 
a limited amount of data in the beginning making the correct estimation of 
parameters like the source position virtually impossible.

It has also taken into account that non-stationary kernels might not be agnostic 
to the structure of the modelled function. Such a covariance function will lead 
to better results if the function matches the structural assumptions of the 
kernel. However, if that is not the case, the results will probably be worse 
than with stationary kernel agnostic to the structure. Especially when modelling 
a plume distribution in a real world scenario one would have to consider 
a multitude of effects like obstacles and local changes in wind.  That might 
make it impossible to derive valid structural assumptions for using 
a non-stationary kernel.

TODO read and maybe include non-stationary papers

\section{Hyper-parameter Selection}
Though Gaussian processes are non-parametric, the choice of the covariance 
function will introduce hyper-parameters $\vc \theta$ (i.\,e.~the length scale) 
which have to be set.  In the following I will discuss three methods for doing 
so.

With the \newterm{test set method} all data available $\mathcal{D}$ will be 
split into two sets $\mathcal{D}_0$ and $\mathcal{D}_*$. The set $\mathcal{D}_0$ 
is used to train Gaussian processes with different covariance functions and 
hyper-parameters. For each model the generalization error $E\ped{G}$ over the 
test set $\mathcal{D}_*$ will be evaluated the parameters with the minimal 
generalization error will be chosen. The error measure can be chosen freely, 
with the root mean square error being a typical choice (see also 
Chapter~\ref{sec:error}).

If the amount of available data is limited, it is common to use 
\newterm{$k$-fold cross validation} where $\mathcal{D}$ is split into $k$ 
disjoint subsets $\mathcal{D}_i$ of equal size and the generalization error will 
be calculated from $k$ models using the respective $\mathcal{D}_i$ as test set 
and the other sets as training data.

The third possibility is to find $\argmax_{\vc \theta} p(\vc\theta | \vc y, \mat 
X, \mathcal{H}_i)$, wherein
\begin{equation}
    p(\vc\theta | \vc y, \mat X, \mathcal{H}_i) = \frac{p(\vc y | \mat X, 
        \vc\theta, \mathcal{H}_i) p(\vc\theta | \mathcal{H}_i)}{p(\vc y | \mat 
        X, \mathcal{H}_i)}
\end{equation}
with marginal likelihood $p(\vc y | \mat X, \vc\theta, \mathcal{H}_i)$, prior 
$p(\vc\theta | \mathcal{H}_i)$, normalization factor $p(\vc y | \mat X, 
\mathcal{H}_i)$, and a set of possible model structures $\mathcal{H}_i$. The 
normalization factor can be difficult to estimate 
\parencite[109]{Rasmussen:2006vz}. For that reason, even though it can more 
easily lead to overfitting, often only the marginal likelihood is optimized 
which is known as \newterm{type~\RN{2} maximum likelihood}. For a Gaussian 
process with $n$ training samples it is given by
\begin{equation}
    \log p(\vc y | \mat X, \vc\theta) = -\frac{1}{2}\del{\vc y\Tr 
        \tilde{\mat{K}}^{-1} \vc y + \log \det \tilde{\mat K} + n \log 2\uppi} 
    \text{.}
\end{equation}
The three summands can be interpreted as the quality of the data fit $\vc y\Tr 
\tilde{\mat K}^{-1} \vc y$, model complexity $\log \det \tilde{\mat K}$, and 
a normalization term $n \log 2\uppi$. Hence, the optimization of the marginal 
likelihood includes an automatic trade-off of model complexity and data fit.

Optimizing the marginal likelihood has the advantage (in comparison to the test 
set method) that a gradient based optimizer can be used. The partial derivatives 
of the likelihood are given by
\begin{equation}
    \dpd{}{\theta_j} \log p(\vc y | \mat X, \vc\theta) = \frac{1}{2} 
    \del{\del{\tilde{\mat K}^{-1}\vc y \vc y\Tr \tilde{\mat K}^{-1} 
            - \tilde{\mat{K}}^{-1}} \dpd{\tilde{\mat K}}{\theta_j}} \text{.}
\end{equation}
Nevertheless, all methods require a complete retraining of the Gaussian process 
as for each update of the hyper-parameters $\tilde{\mat K}^{-1}$ has to be newly 
calculated.  Thus, in an online setting it is far more efficient to keep the 
hyper-parameters fixed or only update them occasionally.

\section{Active Learning}\label{sec:utility}
A setting in which a learning algorithm can freely choose the next training 
input is called \newterm{active learning}. A general introduction to the topic 
is provided by \textcite{Settles:2009vo}. Here, I will focus on how to realize 
active learning in the context of Gaussian processes.

In general, one defines a \newterm{utility} or \newterm{acquisition} function 
$u(\vc x)$ indicating the expected benefit for choosing $\vc x$ as next training 
sample. Hence, the optimal choice is $\argmax_{\vc x} u(\vc x)$. Equivalently, 
it is possible to use the negative of a loss function $u(\vc x) = - \lambda(\vc 
x)$.

The choice of $u(\vc x)$ influences the exploration-exploitation trade-off and 
on which areas of the input space the learning will be focussed. In the 
estimation of a plume distribution samples should be acquired mostly at places 
with high concentrations, but once such an area is well estimated further 
exploration should follow to possibly find further sources.

\Textcite{Marchant:2012wb} proposed the \newtermAbbrev{distance-based upper 
    confidence bound}{DUCB} for a similar scenario of environmental monitoring 
    where ozone concentrations over US territory were to be measured:
\begin{equation}
    u\ped{DUCB}(\vc x) = \mu(\vc x) + s\ped{DUCB}(\vc y) \sbr{\kappa \cdot 
        \sigma^2(\vc x) + \gamma \cdot d(\vc x, \vc x')}
\end{equation}
The mean prediction $\mu(\vc x)$ and the predictive variance $\sigma^2(\vc x)$ 
are obtained directly from the Gaussian process, $d(\vc x, \vc x')$ denotes the 
Euclidean distance of $\vc x$ to the last sample location $\vc x'$. The 
parameter $\kappa$ controls the exploration-exploitation balance. Higher values 
give more importance to decreasing the predictive variance and lead to more 
exploration. The parameter $\gamma \leq 0$ adjusts the distance penalty.  
Favoring location near to the current UAV position might decrease the distance 
travelled and save energy as well as time. The original DUCB formulation by 
\textcite{Marchant:2012wb} did not include the scaling factor $s(\vc y)$. To 
match that formulation it has to be set to $s\ped{DUCB}(\vc y) = 1$. Another 
possible choice will be discussed below.

Though, the ozone concentration scenario appears to be quite similar to the 
plume modelling problem at hand one should note a certain difference. The ozone 
concentration is a quite smooth distribution as \textcite{Marchant:2012wb} used 
the squared exponential covariance function to obtain reasonable results. In 
opposite to that, the spatial distribution of a gas plume is much more localized 
\parencite[this was also noted by][]{Stachniss:2008vz}. Thus, I propose the 
\newtermAbbrev{plume distance-based upper confidence bound}{PDUCB} acquisition 
    function inspired by DUCB, but adjusted:
\begin{multline}
    u\ped{PDUCB}(\vc x) = \del{1 - a} \cdot \ln\del{\mu_+(\vc x) + \varepsilon} 
    + a \cdot \ln \varepsilon \\ + s\ped{PDUCB}(\vc y) \sbr{\kappa \cdot 
        \sigma^2(\vc x) + \gamma \cdot d^2(\vc x, \vc x')}
\end{multline}
with
\begin{align}
    a &= \e^{-\mu_+(\vc x) / \varepsilon} \\
    \mu_+(\vc x) &= \max\cbr{0, \mu(\vc x)}
\end{align}

Using the logarithm of the prediction mean makes this utility function sensitive 
for small concentration changes in areas of low concentration. These can hint 
towards areas with higher concentration.  Being sensitive to the same absolute 
change for high concentrations is not as important. As the concentration might 
be equal to zero strict positiveness has to be explicitly ensured by a small 
$\varepsilon > 0$.  Also, semi-positiveness of the predictive mean has to be 
ensured.

It is desirable to have differentiable acquisition functions to be able to use 
gradient based optimizers. However, due to the logarithm the function would not 
be differentiable for $\mu(\vc x) \rightarrow 0$. Thus, it is weighted with $(1 
- a)$ and faded out to $\ln \varepsilon$ to restore differentiability (proof in 
Appendix~\ref{sec:pducb-diff}).

A further change in PDUCB compared to DUCB is the squaring of the distance which 
will reduce the penalty around $\vc x'$ (while increasing it further away).  
This should be advantageous as the unsquared distance function tends to force 
$\vc x$ much closer to $\vc x'$. Though the next sample should be near to $\vc 
x'$, it should not be too close to $\vc x'$.  Otherwise, not much new 
information would be gained due to the spatial correlation.

Apart from some explicit values, \textcite{Marchant:2012wb} do not discuss how 
to choose the parameters $\kappa$ and $\gamma$. Nevertheless, some observations 
can be made for both DUCB and PDUCB\@. Firstly, one should choose $\kappa \cdot 
\max \sigma^2(\vc x) > \max \mu(\vc x)$. Otherwise, one can get stuck in a local 
maximum as the mean prediction term might get larger than the predictive 
variance term anywhere in the input space. Even though it can be a good strategy 
to exploit maxima first, exploration should continue once the distribution 
around the maximum is accurately known. A too small $\kappa$ is also problematic 
as it prevents any exploitation. Secondly, $\abs{\gamma}$ should not be too 
large or the distance penalty would dominate and also lead to one getting stuck 
in one position $\vc x = \vc x'$. Thirdly, $\varepsilon$ influences the 
sensitivity for low concentrations and should therefore be small, but large 
enough to prevent numerical problems in the evaluation of the logarithm. A value 
of $\varepsilon = 10^{-30}$ seems to work well (see Chapter~\ref{sec:exp}). 
Finally, it should be noted that PDUCB needs a different scaling 
$s\ped{PDUCB}(\vc y)$ because the logarithm of the mean prediction is used. 
Assuming $\mu(\vc x)$ will be in the range \numrange{0}{1}, the range of the 
logarithmic mean prediction term will be $\ln(1) - \ln(\varepsilon) \approx 70$. 
Thus, setting $s\ped{PDUCB}(\vc y) = 70$ will make the other parameters of DUCB 
and PDUCB roughly comparable.

Typically, one knows the spatial dimensions of the input space which allows to 
estimate a reasonable $\gamma$ in relation to $\kappa$ as the maximal predictive 
variance is given by $\sigma\ped{n} + \sigma\ped{k}$.  However, the maximal $\vc 
y$ determining $\max \mu(\vc x)$ which is important for the absolute values of 
$\gamma$ and $\kappa$ in relation to the prediction mean might not be known in 
advance.  Thus, it would be helpful to set these parameters automatically from 
$\vc y$, the data seen so far.  This can be done by setting the scaling factor 
$s(\vc y)$ defined for the respective acquisition functions as
\begin{align}
    s\ped{DUCB}(\vc y) &= \max \vc y \label{eqn:scale-ducb}\\
    s\ped{PDUCB}(\vc y) &= \ln(\max \vc y + \varepsilon) - \ln \varepsilon 
    \text{.} \label{eqn:scale-pducb}
\end{align}
The parameters $\kappa$ and $\gamma$ can then be set independent of the actual 
values of $\vc y$.

A third potential acquisition function, also balancing exploration and 
exploitation, can be derived from the work by \textcite{Osborne:2009tn}. They 
introduce a Bayesian approach for global optimization. Adopting their approach 
for a one-step look-ahead one first defines a loss function equal to the 
negative maximum after making a new observation
\begin{equation}
    \lambda\ped{GO} (y_*) = \left\{ \begin{array}{ll}-y_* & y_* > \eta \\ -\eta 
            & y_* \leq \eta \end{array} \right.
\end{equation}
with $\eta = \max \vc y$. From this the expected loss can be determined as
\begin{equation}\begin{split}
    \varLambda\ped{GO}(\vc x) &= \int \lambda\ped{GO}(y_*) p(y_* | \vc x, X, \vc 
    y) \dif y_* \\
    &= -\eta - \del{\mu(\vc x) - \eta} \varPhi\!\del{\eta; \mu(\vc x), 
        \sigma^2(\vc x)} - \sigma^2(\vc x) N\!\del{\eta; \mu(\vc x), 
        \sigma^2(\vc x)}
    \text{.}
\end{split}\end{equation}
In this equation $N(x; \mu, \sigma^2)$ and $\varPhi(x; \mu, \sigma^2)$ are the 
Gaussian probability density and respectively the Gaussian cumulative 
distribution function with mean $\mu$ and variance $\sigma^2$. Adding a distance 
penalty term to the negative of $\varLambda\ped{GO}(\vc x)$ gives the final 
utility function
\begin{align}\begin{split}
    u\ped{GO}(\vc x) &= -\varLambda\ped{GO}(\vc x) + \gamma \cdot d^2(\vc x, \vc 
    x') \\
    &= \eta + \del{\mu(\vc x) - \eta} \varPhi\!\del{\eta; \mu(\vc x), 
        \sigma^2(\vc x)} + \sigma^2(\vc x) N\!\del{\eta; \mu(\vc x), 
        \sigma^2(\vc x)}\\ &\quad+ \gamma \cdot d^2(\vc x, \vc x') \text{.}
\end{split} \end{align}
TODO include multi-step look-ahead? process mixtures?

\begin{figure}
    \centering
    \includegraphics{plots/acqfns}
    \caption[Visualization of acquisition functions]{Visualization of 
    acquisition functions. The columns of the plot matrix correspond to the 
    three different acquisition functions (DUCB, PDUCB, GO) and the rows show 
    the state after 3, 10, and 
        20 iterations.  The utility functions were normalized by dividing by 
           $\max_{x \in \intcc{-15, 15}} \abs[0]{u(x)}$.
        The parameters used for these plots were $\kappa = 1.25$, $\gamma 
        = -0.0002$, $\varepsilon = 10^{-30}$, $s\ped{DUCB}(\vc y) = 1$, and 
        $s\ped{PDUCB}(\vc y) = 70$. The initial sample was always chosen at $x_0 
        = -7$.}\label{fig:acqfns}
\end{figure}
In Figure~\ref{fig:acqfns} a graphical comparison of the proposed utility 
functions is given for a one-dimensional example. It can be seen that DUCB 
heavily focusses on the function maximum without much exploration in other 
areas. The GO acquisition function also acquires more samples around the 
maximum, but reduces the uncertainty more equally over the domain. In comparison 
to these two functions PDUCB seems to also focus around the maxima, but with 
a wider exploration around those. In Chapter~\ref{sec:cmputility} I will more 
closely look on how well these functions work to estimate a plume dispersion 
with different parameter choices.

\subsection{Multiple UAVs}\label{sec:multiple-uavs}
Using more than one UAV might considerably speed up the estimation of a plume 
distribution. For that reason I propose a method to extend the target selection 
with an acquisition function $u(\vc x)$ to multiple UAVs. To not impair the 
performance of the original utility function one UAV will be selected as 
a \newterm{master UAV} and uses exactly $u(\vc x)$. Without loss of generality 
we can assign the index $i = 1$ to that UAV\@. For all other UAVs with $i > 1$ 
the following modified acquisition function
\begin{equation}
    u_i(\vc x) = u(\vc x) + \rho \sum_{j = 1,\ i \neq j}^n d^2(\vc x, \vc x_j)
\end{equation}
will be used where $\vc x_j$ is the position of the $j$-th UAV\@. This modified 
acquisition function basically introduces a penalty for locations close to other 
UAVs to spread them out.

\subsection{Bootstrapping}\label{sec:bootstrapping}
As long as no minimal concentration has been measured all of the proposed 
acquisition functions do not have a unique maximum. Though one could choose 
sampling locations randomly, this might take a long time until the plume gets 
discovered.  Hence, it is best to employ a more systematic search strategy in 
the beginning.

Here, three variations will be used. First, surrounding the area at a medium 
height.  Without noise this is sufficient to obtain enough information for 
a successful usage of the discussed acquisition functions.

Considering noise a second strategy is needed as measurements of low 
concentrations are not reliably anymore. This strategy consists of surrounding 
the area at different heights until a criterion that a plume as been found is 
fulfilled. The criterion employed here is that the maximum of all concentration 
measurements $\max y_i$ for one surrounding has to be larger than $5\sigma(y)$ 
with $\sigma(y)$ being the standard deviation of the $y_i$.

This second strategy may also take a long time before a plume has been found, 
but should be faster than random exploration. If the wind direction is known 
(which is not the case in the standard QRSim scenarios), a third strategy can be 
used.  It improves the second strategy by not doing complete surrounds of the 
area, but only along the two area edges which are ``hit'' by the wind. Those are 
the only two were the plume might be detected.

These bootstrapping strategies can be easily extended to multiple UAVs by 
assigning different heights to each one.

Most of the samples acquired during this bootstrapping process do not improve 
the plume distribution. Thus, most of them were discarded to improve efficiency.  
Only the measurements from surrounding the area for the last time before the 
plume was identified were used for the training of the Gaussian process.


\chapter{Error Measures}\label{sec:error}
To be able to compare different statistical models some kind of performance 
measure, usually in the form of an error measure, is needed. In the plume 
modeling task one is interested in the deviation of the predicted mean 
concentrations $\mu(\vc x)$ from the true ones $c(\vc x)$. When using the $L^2$ 
norm and integrating over the complete task volume $V$ the \newtermAbbrev{root 
    mean integrated square error}{RMISE}
\begin{equation}
    E\ped{RMISE} = \sqrt{\frac{1}{v} \int_V \del{c(\vc x) - \mu(\vc x)}^2 
        \dif\vc x}
\end{equation}
with
\begin{equation}
    v = \int_V \dif\vc x
\end{equation}
is obtained. However, it can be assumed that accurate predictions are more 
important where the concentration is actually high 
\parencite[cp.][]{Marchant:2012wb}. Thus, it might be beneficial to introduce 
a weighting factor $w(\vc x)$ to give the \newtermAbbrev{weighted root mean 
    integrated square error}{WRMISE}
\begin{equation}
    E\ped{WRMISE} = \sqrt{\frac{1}{v} \int_V \del{c(\vc x) - \mu(\vc x)}^2 w(\vc 
        x) \dif\vc x} \text{,}
\end{equation}
with
\begin{equation}
    w(\vc x) = \frac{c(\vc x) - \min_{\vc x'} c(\vc x')}{\max_{\vc x'} c(\vc x') 
        - \min_{\vc x'} c(\vc x')} \text{.}
\end{equation}
Note that in areas with concentration of almost or even exactly zero the WRMISE 
will always be close to zero and thus allowing the model to make highly 
inaccurate predictions. Therefore, the WRMISE should not be used as sole measure 
in plume modeling. Using both errors it is possible to ensure good overall fit 
of the prediction without large inaccuracies and to compare which model has the 
better fit in the interesting areas.

Unfortunately, the RMISE and WRMISE cannot easily be calculated analytically and 
one has to restrain to approximating the integral from a finite set $\{\vc x_i 
| i = 1, \dots, n\}$ of samples. If the $x_i$ are distributed according to the 
probability density function $p(\vc x)$, the approximation has the form
\begin{align}
    \hat E\ped{RMISE} &= \sqrt{\frac{1}{vZ} \sum_{i=1}^n \frac{\del{c(\vc x_i) 
                - \mu(\vc x_i)}^2}{p(\vc x_i)}} \\
    \hat E\ped{WRMISE} &= \sqrt{\frac{1}{vZ} \sum_{i=1}^n \frac{\del{c(\vc x_i) 
                - \mu(\vc x_i)}^2 w(\vc x_i)}{p(\vc x_i)}} \text{.}
\end{align}
The normalization constant $Z$ is given by
\begin{equation}
    Z = \sum_{i=1}^n \frac{1}{p(\vc x_i)} \text{.}
\end{equation}

The probability density $p(\vc x)$ can be estimated using Gaussian 
\abbrev{kernel density estimation}{KDE}. There exist different methods to 
determine the bandwidth parameter of the KDE\@. In this work Scott's Rule 
\parencite{Scott:2009tl} was used.

\section{Selecting Samples for Error Approximation}\label{sec:mh}
Up to now it is still open how to select the $\vc x_i$ for approximating the 
error measures.  Using a regularly spaced grid, it is either likely to miss the 
important areas as the plume is relatively localized, or it is so fine grained 
that the evaluation of the error takes a long time. Thus, I used a slightly 
modified \abbrev{Metropolis-Hastings}{MH} algorithm to accumulate samples in 
areas with a high concentration.

The standard Metropolis-Hastings \parencite{Chib:1994ud} is used to sample from 
a probability distribution $p(\vc x)$ given only a function $f(\vc x)$ 
proportional to $p(\vc x)$. It starts at a random location $\vc x_1$.  Then in 
each iteration $i > 1$ a new candidate location $\vc x_*$ is picked from 
a symmetric\footnote{$Q(\vc a | \vc b) = Q(\vc b | \vc a)$} proposal 
distribution and an acceptance ratio $\xi = f(\vc x_*) / f(\vc x_i)$ is 
calculated. The candidate $\vc x_*$ is accepted as $\vc x_{i + 1} = \vc x_*$ 
with probability $\xi$ ($\xi \geq 1$ automatically accepts). If it is rejected, 
$\vc x_{i + 1} = \vc x_i$ will be used.

To select the samples for the error approximation this algorithm can be used 
with the true concentrations $f(\vc x) = c(\vc x)$. For this purpose it is not 
necessary that the samples exactly follow a specific probability distribution.  
That makes it possible to make a few adoptions for better results.

First of all, $\vc x_*$ can always be added to the set of samples independent of 
acceptance or rejection instead of the accepted sample.  This makes each 
location unique (as long as the same location does not get proposed twice).  
Having multiple instance of the same location within the samples would not 
increase the accuracy of the error approximation. Also, this leads to a few more 
samples towards the concentration tails where the concentration is already low 
but spatially close to high concentrations.  Thus, the approximation around the 
concentration slope can be expected to be better.

A further change concerns the initial location $\vc x_1$. Choosing at random 
might place it in an area with zero concentration which renders the acceptance 
ratio $\xi$ undefined. Hence, it is better to base the initial location near 
a source location. A placement directly at the source location is not possible 
given that a plume dispersion (Equation~\ref{eqn:plumedisp}) is undefined that 
place.

In case of multiple sources the sampling should be started from each source 
location and the resulting sets should be joined. The concentration between 
sources can be rather low making it unlikely to switch from one plume to 
another. Also, the Metropolis-Hastings algorithm is unlikely to sample in low 
concentration areas, especially in some distance to the plumes. Thus, a number 
of uniformly sampled locations should be added.

To ``smooth'' out the samples around the plumes even more one can use only every 
$k$-th sample of the Metropolis-Hastings algorithm and use the remaining samples 
as center of Gaussian distributions and draw $k$ more samples from each of 
these.

\section{QRSim Reward}\label{sec:qrsim-reward}
The plume modeling scenarios in \textcite{denardi2013rn} themselves define 
a reward
\begin{equation}
    R = - \sum_{i = 1}^n \del{c(\vc x) - \mu(\vc x)}^2
\end{equation}
as a performance measure. This is essentially the negative of 
$\hat{E}\ped{RMISE}^2$ without normalization for the sampling density. Thus, 
with a non-uniform sampling this reward will give a biased estimate attributing 
more importance to areas with more sample locations.

The set of $\vc x_i$ used to calculate the reward in QRSim consists ouf of two 
parts. One-fifth is sampled uniformly over the whole volume, whereas the 
remaining samples are taken uniformly from the areas where the concentration 
exceeds the limit $c_{\min} = Q \cdot 10^{-3} \cdot 
\si{\second\per\meter\cubed}$.

The bias introduced by that towards the interesting regions is not necessarily 
bad. The same is done in the WRMISE\@. Unfortunately, the sampling strategy will 
not sample at the low side of a (steep) concentration boundary and will give 
a bad estimate of the reward around those boundaries. In contrast to that, the 
Metropolis-Hastings based sampling algorithm acquires also some samples in the 
low concentration area around a high concentration.  This difference is 
visualized in Figure~\ref{fig:err-sampling}.

\begin{figure}
    \centering
    \includegraphics{plots/err-sampling}
    \caption[Comparison of error estimation sampling methods]{One-dimensional 
        comparison of the Metropolis-Hastings based sampling and the QRSim 
        sampling for error estimation. For both approaches 60 samples with 
        regard to the plotted Gaussian density were taken excluding the 
        uniformly distributed samples.  The $y$ position of the scatter marks 
        has no meaning in this plot. For the MH based sampling in this plot 
        every fifth sample was used.  A Gaussian was used as proposal 
        distribution with $\sigma = 2$.  The same distribution was used to 
        create five additional samples for each MH based 
        sample.}\label{fig:err-sampling}
\end{figure}

\section{Normalized Error}
For a reliable comparison of the models it is not sufficient to compare them 
based on a single trial. A good model should provide a low error in different 
trials.

In the plume modeling scenarios the concentration density and spatial extent of 
the plume varies. The error measures directly depend upon the absolute 
concentration values and, therefore, also depend on the specific trial.

Directly averaging the error over trials would give a higher weighting to trials 
with high concentration densities or a large plume extent. A good plume 
modeling method should adapt to the concentration levels and be more precise 
for lower overall concentrations. To achieve a fair weighting of the trials the 
error can be normalized as
\begin{equation}
    F = \frac{E}{E_0}
\end{equation}
for each trial, where $E$ is the current error estimate and $E_0$ the error 
estimate for an all zero prediction. The error estimate can be freely chosen 
with regard to its properties. Note that this equation can also be applied to 
the reward $R$ as error measure as the minus cancels out. When averaging $F$ 
over trials each trial is weighted equally.

The normalized error $F$ has also the advantage of being readily interpretable.  
For $F > 1$ the prediction is worse than an all zero prediction.  This should 
not happen with a viable modeling method. For $F < 1$ the prediction is better 
than an all zero prediction and the error has been decreased by $(1 - F) \cdot 
\SI{100}{\percent}$.

\chapter{Technical Details}\label{sec:tech}
To implement the discussed algorithms in a functioning system some further 
details have to be taken in consideration. I will discuss these in the following 
after giving short general overview of the implementation.

\section{Implementation}
All simulations were performed with QRSim \parencite{denardi2013rn}, a quadrotor 
simulator developed specifically to test high level tasks. It supports multiple 
UAVs which are simulated with realistic dynamics of the platform. Also, equipped 
sensors (e.\,g.~GPS, IMU) are simulated with different sources of inaccuracies.  
This includes wind influencing the vehicles as well as the plume dispersions.

I decided to implement the algorithms (i.\,e.~Gaussian processes, acquisition 
functions) in Python because of its high-level programming constructs and 
excellent scientific computing support with NumPy and SciPy 
\parencite{Oliphant:2007dm}. Communication between the Python part and QRSim 
implemented in MATLAB were done with a TCP interface based on the Google 
protocol 
buffers\footnote{\url{https://developers.google.com/protocol-buffers/}}.

Though there exist several Gaussian process implementations for Python like for 
example Scikit-learn \parencite[i.\,e.][]{scikit-learn}, none supports online 
updates to my knowledge.  For that reason I developed an own implementation.

\section{Function optimization}\label{sec:fnopt}
For finding the maximum of an acquisition function the SciPy wrapper of the 
\mbox{FORTRAN} implementation of the \mbox{L-BFGS-B} algorithm 
\parencite{Byrd:2006iv, Zhu:1997br} was used.  As gradient based optimizer with 
the possibility to constrain the search space to the task volume it is well 
suited for the task.

To choose the starting location $\vc x_0$ the utility function was evaluated on 
a coarse $5 \times 5 \times 5$ grid and the location with the maximal value was 
chosen. The convergence parameters were set to $\mathit{pgtol} = 10^{-10}$ and 
$\mathit{factr} = 100$. The rather flat DUCB gradient required this strict 
settings. For the other acquisition functions a good convergence was also 
possible with higher values (i.\,e.~$\mathit{pgtol} = 10^{-5}$ and $\mathit{factr} 
= 10^7$).  Nevertheless, the same parameters were used for all utility 
functions.

Noisy plume measurements produce a large number of local maxima in the utility 
function. Hence, in those scenarios the optimization has to be performed 
multiple times using some additional starting points. Those were chosen from 
Gaussian distributions with a standard deviation of \SI{5}{\meter} centered on 
the current UAV position and the location of the maximal recorded plume 
measurement. From each of these Gaussian five additional starting locations for 
the optimization were sampled.

\section{UAV Control}
The QRSim simulator accepts different UAV control commands including way-points 
and target velocities. Setting directly the way-points would be natural as the 
optimization of the acquisition function leads to target coordinates. However, 
the QRSim way-point controller did not proof to be very reliable and UAVs 
leaving the simulation area were not uncommon. To circumvent this, a different 
algorithm was used. Each way-point $\vc t_i$ was translated to velocity commands 
$\vc v_i$ sent to QRSim for the $i$-th out of $n$ UAVs with
\begin{align}
        \vc v_i &= \del{\begin{array}{c}
            d_{i, 1} \min\{1, v_{\max,1} / d_{i,\mathrm{hor}}\} \\
            d_{i, 2} \min\{1, v_{\max,2} / d_{i,\mathrm{hor}}\} \\
            \min\{v_{\max,3}, \max\{-v_{\max,3}, d_{i, 3}\}\}
        \end{array}} \label{eqn:final_velocities} \\
        d_{i,\mathrm{hor}} &= \sqrt{d_{i, 1}^2 + d_{i, 2}^2} \\
    \begin{split}
        \vc d_i &= \del{d_{i,1}, d_{i,2}, d_{i,3}}\Tr \\
        &= \diag\!\del{\vc v\ped{\max}} \del{u_1 \del{\vc t_i - \vc x_i} + u_2 
            \sum_{j = 1,\ i \neq j}^{n} \frac{\vc x_i - \vc x_j}{\abs{\vc x_i 
                    - \vc x_j}^3}}
    \end{split}
\end{align}
where $\vc x_i$ are the current UAV positions, $u_1 = \SI{0.025}{\per\meter}$ 
and $u_2 = \SI{5}{\meter\squared}$ are scaling constants, and $\vc v_{\max} 
= (v_{\max,1}, v_{\max,2}, v_{\max,3})\Tr = (\SI{6}{\meter\per\second}, 
\SI{6}{\meter\per\second}, \SI{6}{\meter\per\second})\Tr$ a vector of speed 
limits\footnote{QRSim additionally applies its own speed limit independently per 
    direction in the UAV body frame of reference. It is 
    \SI{3}{\meter\per\second} for the two horizontal axes.}.  The $\vc d_i$ 
vector components consist of two terms.  One directs the speed towards the 
target proportional to the remaining distance resulting in a slow down near to 
the target. The remaining term acts like a repelling force between the UAVs to 
keep them from colliding.  It is proportional to the inverse of the square of 
the distance. The power of three occurs because the direction vector $\vc x_i 
- \vc x_j$ has to be normalized.  With Equation~\ref{eqn:final_velocities} the 
velocity will be limited while keeping the overall horizontal direction.

To prevent the UAVs from going astray a safety margin of \SI{10}{\meter} is 
defined at the boundaries of the simulated volume. When an UAV enters this 
margin in one dimension the corresponding velocity component will be set to $\pm 
\vc v_{\max}$.

A target $\vc t_i$ is considered to be reached when $\abs{\vc t_i - x_i} 
< \SI{3}{\meter}$.

\chapter{Evaluation and Simulation Experiments}\label{sec:exp}
A number of simulation experiments has been performed to evaluate the proposed 
methods for modelling plume distributions. First, the most suitable covariance 
function and hyper-parameters were obtained. These results were then used to 
compare the different acquisition functions based on the noise-free scenarios.  
Subsequently, the performance of PDUCB was evaluated on the noisy scenarios and 
using multiple UAVs.

\section{Best Covariance Function for Plume Modelling}\label{sec:bestkernel}
To obtain the best covariance function including its parameters to approximate 
a plume distribution these were evaluated using the test-set method. For each of 
the single source Gaussian (G-NF-SS-SV), the single source dispersion 
(D-NF-SS-SV), and the multiple source dispersion (D-NF-MS-SV), all without 
noise, 50 random instances were created. For each instance a set sampling 
locations was generated using the Metropolis-Hastings based technique described 
in Chapter~\ref{sec:mh}. Herein, every fifth Metropolis-Hastings sample was used 
in the final set and was used as mean of Gaussian with a standard deviation 
$\sigma = \SI{6}{\meter}$ to draw five more samples to include in the final set.  
The proposal distribution of the Metropolis-Hastings algorithm was also 
a Gaussian with standard deviation $\sigma = \SI{6}{\meter}$. In addition, 1000 
uniformly samples were added to the final set of samples. All samples outside of 
the scenario volume were dismissed. From all kept sample points 1000 were 
randomly selected for training and the rest was used as test set to determine 
the error.

Obtaining the training samples in this way should roughly mirror a good sampling 
with an UAV with many samples in the areas of high concentration and a few in 
the remaining areas. The advantage using this way of sampling is that it allows 
us to test different kernels independently on the exact behavior of the UAV and 
time consuming simulation of it.

The kernels tested were the squared exponential, the Mat\'ern kernel with $\nu 
= 5/2$, the Mat\'ern kernel with $\nu = 3/2$, and the exponential kernel. The 
length scales tested ranged from \SI{1}{\meter} to \SI{100}{\meter}. The process 
variance was fixed as $\sigma\ped{k}^2 = 1$. Note that this parameter has no 
effect on the predictive mean as long as the assumed noise variance 
$\sigma\ped{n}^2$ is zero.

The average of the fraction of the remaining error is plotted in 
Figure~\ref{fig:lengthscales} for the different kernels and error measures. The 
minimum is roughly the same for all kernels and lies around $\ell 
= \SI{5}{\meter}$.  However, the behavior differs considerably for non-optimal 
length scales.  The smoother (the more often the kernel is differentiable) the 
more the error increases for too large length scales.  Especially, for the 
squared exponential covariance function this increase is quite abrupt. Only for 
very large length scales it decreases again for the squared exponential kernel.

\begin{figure}
    \centering
    \includegraphics{plots/lengthscales}
    \caption[Influence of the length scale of the covariance functions]{The 
        average remaining fraction of the inital error for different covariance 
        function in dependence of length scale.  The rows correspond to the 
        RMISE, WRMISE, and QRSim reward error measures.  The columns correspond 
        to a single source Gaussian (G-NF-SS), a single source Gaussian 
        dispersion (D-NF-SS), and a multiple source Gaussian dispersion 
        (D-NF-MS). All scenarios were simulated without sensor noise.  Error 
        bars represent the standard error. The boundary of $1.0$ where the error 
        of the trained Gaussian process is larger than an all zero prediction is 
        marked with a horizontal line.}\label{fig:lengthscales}
\end{figure}

Comparing the WRMISE to the RMISE the former one stays quite low even for larger 
length scales. This indicates that in the area of the plume (also due to the more 
dense sampling) a good fit is still obtained, but around that area the 
prediction gets worse. Thus, the steep concentration gradients around the plume 
are not well captured in that case.

The results give also an idea how good of a fit can be expected at best when 
using an UAV\@. Whereas the fraction of the remaining error decreases to nearly 
zero for the single source Gaussian, it stays above 0.6 for the dispersion 
scenario with the more localized plume distribution. The reward error measure is 
decreased to lower levels, but this is likely to underestimation of the error at 
the plume boundaries as argued in Chapter~\ref{sec:qrsim-reward}.

Besides the error measures the log likelihood of each trained Gaussian process 
was calculated. In Figure~\ref{fig:loglikelihood} the average over trials is 
plotted. Only for the squared exponential kernel the maximum of the log 
likelihood corresponds to the minimum of the RMISE\@. Towards longer length 
scales the likelihood declines very steeply. Using the log likelihood to 
estimate the length scales for the other covariance functions would largely 
overestimate it.

\begin{figure}
    \centering
    \includegraphics{plots/loglikelihood}
    \caption[Log likelihood in dependence of the kernel lengthscale]{The average 
        log likelihood of the training data in dependence of the length scale 
        using different kernels.  Each of the three plots shows one noise free 
        scenario of the single source Gaussian (G-NF-SS), single source Gaussian 
        dispersion (D-NF-SS), and multiple source Gaussian dispersion (D-NF-MS).  
        Error bars represent the standard error.}\label{fig:loglikelihood}
\end{figure}

Taken these results together it is best to choose a non-smooth kernel with 
a length scale of $\ell = \SI{5}{\meter}$. As it is advantageous to be able to 
use a gradient based optimizer for the optimization of acquisition functions, 
I decided to use the Matérn kernel with $\nu = 3/2$ in the further experiments, 
which gives a once mean square differentiable Gaussian process in opposite to 
the exponential kernel. Unfortunately, optimizing the length scale using the 
likelihood would not give good results and I fixed the length scale at $\ell 
= \SI{5}{\meter}$. Also, including a prior in the log likelihood does not help 
here. In example to shift the maximum of the likelihood for the chosen kernel to 
\SI{5}{\meter} a Gaussian prior would need a standard deviation of less then 
$\sigma_{\ell} < \e^{-2078} / \sqrt{2\uppi} \approx 0$ (see 
Apendix~\ref{sec:prior}).  Thus, effectively resulting in a fixed length scale.

\section{Comparison of Utility Functions}\label{sec:cmputility}
Given the kernel chosen in the previous section I continued to compare the 
different utility functions in the noiseless scenarios single source Gaussian 
(G-NF-SS-SV), single source dispersion (D-NF-SS-SV), and multiple source 
dispersion (D-NF-MS-SV).

For each given scenario 20~trials were performed. In each run the UAV first 
surrounded the simulation area in a height of \SI{40}{\meter} with a margin of 
\SI{10}{\meter} to the boundaries of the simulated volume. After that further 
way-points were chosen with one of the acquisition functions discussed in 
Chapter~\ref{sec:utility}. The optimization of that functions has been described 
in Chapter~\ref{sec:fnopt}.

Each trial was allowed to run for a maximum of \SI{3000}{\second} in simulation 
time. However, when a new target way-point was within \SI{3}{\meter} of the 
previous one the UAV was considered to become stuck in a maximum of the 
acquisition function and the simulation was stopped at that point to reduce 
overall simulation time. A plume measurement was taken every second.

The error measures were estimated as described in Chapter~\ref{sec:error}. The 
sampling locations for that where chosen as 1000 uniformly distributed sampling 
locations, every tenth of 4200 locations from the Metropolis-Hastings algorithm 
with Gaussian proposal distribution with standard deviation $\sigma 
= \SI{10}{\meter}$, and 10 more locations sampled from the proposal distribution 
for each of included Metropolis-Hastings samples.

I tested all three utility functions proposed in Chapter~\ref{sec:utility}: 
DUCB, PDUCB (with $\varepsilon = 10^{-30}$), and GO\@. DUCB was tested with 
a constant scaling factor of $s\ped{DUCB}(\vc y) = 1$ and the automatic scaling 
in Equation~\ref{eqn:scale-ducb}.  PDUCB was tested with a constant scaling 
factor of $s\ped{PDUCB}(\vc y) = 70$ (a little bit more than $-\log 
\varepsilon$) and the automatic scaling in Equation~\ref{eqn:scale-pducb}.  
Furthermore, I performed a parameter search over $\kappa \in \cbr{0.1, 0.5, 
0.75, 1, 1.25, 1.5, 2}$ and $\gamma \in \cbr{0} \cup \cbr{-10^p | p = -9, -8, 
  \dots, -2}$. Note that for the GO utility function the $\kappa$ parameter has 
no effect.

Figure~\ref{fig:psearch-G-NF-SS-SV}--\ref{fig:psearch-D-NF-MS-SV} visualize the 
normalized error for the different scenarios.  The respective parameters and 
values of the minima (excluding the QRSim reward) are listed in 
Table~\ref{tbl:err-g-nf-ss-sv}--\ref{tbl:err-d-nf-ms-sv}.  The average reduction 
(over trials) of the RMISE against simulation time in the single source Gaussian 
scenario (G-NF-SS-SV) is plotted in Figure~\ref{fig:errtrace-nf} and looks 
essentially the same for WRMISE and the QRSim reward and therefore it is not 
shown fer those measures. Finally, Figure~TODO visualizes some example UAV 
trajectories for the different acquisition functions.

\begin{figure}
    \centering
    \includegraphics{plots/psearch-G-NF-SS-SV}
    \caption[Remaining fraction of the initial error (G-NF-SS-SV)]{The average 
        remaining fraction of the initial error for different measures, utility 
        functions, parameters in the noiseless single source Gaussian scenario 
        (G-NF-SS-SV).  The columns represent the RMISE, WRMISE, and QRSim reward 
        error measure.  The rows represent the DUCB, auto-scaled DUCB, PDUCB, 
        auto-scaled PDUCB and GO utility functions. The auto-scaled versions use 
        the scaling factor defined in Equations~\ref{eqn:scale-ducb} 
        and~\ref{eqn:scale-pducb}, in contrast to a constant scaling factor. The 
        minimum of each plot is marked with 
        cross.}\label{fig:psearch-G-NF-SS-SV}
\end{figure}

\begin{figure}
    \centering
    \includegraphics{plots/psearch-D-NF-SS-SV}
    \caption[Remaining fraction of the initial error (D-NF-SS-SV)]{The average 
        remaining fraction of the initial error for different measures, utility 
        functions, parameters in the noiseless single source Gaussian dispersion 
        scenario (D-NF-SS-SV).  The columns represent the RMISE, WRMISE, and 
        QRSim reward error measure.  The rows represent the DUCB, auto-scaled 
        DUCB, PDUCB, auto-scaled PDUCB and GO utility functions. The auto-scaled 
        versions use the scaling factor defined in 
        Equations~\ref{eqn:scale-ducb} and~\ref{eqn:scale-pducb}, in contrast to 
        a constant scaling factor.  The minimum of each plot is marked with 
        cross.}\label{fig:psearch-D-NF-SS-SV}
\end{figure}

\begin{figure}
    \centering
    \includegraphics{plots/psearch-D-NF-MS-SV}
    \caption[Remaining fraction of the initial error (D-NF-MS-SV)]{The average 
        remaining fraction of the initial error for different measures, utility 
        functions, parameters in the noiseless multiple source Gaussian 
        dispersion scenario (D-NF-SS-SV).  The columns represent the RMISE, 
        WRMISE, and QRSim reward error measure.  The rows represent the DUCB, 
        auto-scaled DUCB, PDUCB, auto-scaled PDUCB and GO utility functions. The 
        auto-scaled versions use the scaling factor defined in 
        Equations~\ref{eqn:scale-ducb} and~\ref{eqn:scale-pducb}, in contrast to 
        a constant scaling factor.  The minimum of each plot is marked with 
        cross.}\label{fig:psearch-D-NF-MS-SV}
\end{figure}

\newenvironment{errtbl}{\begin{tabular}{lllSSSS}\toprule}{\bottomrule\end{tabular}}
\newcommand*{\errtblhead}[1]{
        & & &
        \multicolumn{2}{c}{#1} &
        \multicolumn{2}{c}{Norm.\ #1} \\
        \cmidrule(lr){4-5} \cmidrule(lr){6-7}

        Utility function &
        \multicolumn{1}{l}{$\kappa$} &
        \multicolumn{1}{l}{$\gamma$} &
        \multicolumn{1}{c}{Mean} &
        \multicolumn{1}{c}{Std} &
        \multicolumn{1}{c}{Mean} &
        \multicolumn{1}{c}{Std} \\
        & & &
        \multicolumn{1}{c}{\si{\nano\gram\per\meter\cubed}} &
        \multicolumn{1}{c}{\si{\nano\gram\per\meter\cubed}} &
        & \\ \midrule }

\begin{table}
    \centering
    \begin{errtbl}
        \errtblhead{RMISE}
        DUCB & 1.25 & \num{-1e-07} & 249.47 & 154.87 & 0.29 & 0.20 \\
        auto-scaled DUCB & 1.50 & \num{-1e-08} & 250.56 & 154.55 & 0.31 & 0.25 
        \\
        PDUCB & 0.10 & \num{-1e-09} & 191.16 & 198.60 & 0.18 & 0.13 \\
        auto-scaled PDUCB & 0.10 & \num{-1e-07} & 190.73 & 180.43 & 0.18 & 0.12 \\
        GO & 0.10 & \num{-1e-08} & 745.74 & 317.59 & 0.80 & 0.13 \\
        \midrule
        \\
        \errtblhead{WRMISE}
        DUCB & 1.25 & \num{-1e-07} & 115.25 & 109.04 & 0.21 & 0.22 \\
        auto-scaled DUCB & 1.50 & \num{-1e-08} & 122.45 & 112.15 & 0.25 & 0.28 
        \\
        PDUCB & 0.10 & \num{-1e-06} & 91.73 & 122.39 & 0.13 & 0.14 \\
        auto-scaled PDUCB & 0.10 & \num{-1e-07} & 92.62 & 120.08 & 0.13 & 0.13 \\
        GO & 0.10 & \num{-1e-08} & 472.51 & 219.39 & 0.80 & 0.15 \\
    \end{errtbl}
    \caption[Minimal error values G-NF-SS-SV.]{The minimal obtained error (RMISE 
        and WRMISE) for each acquisition function and the parameter values used 
        in the single source Gaussian scenario 
        (G-NF-SS-SV).}\label{tbl:err-g-nf-ss-sv}
\end{table}

\begin{table}
    \centering
    \begin{errtbl}
        \errtblhead{RMISE}
        DUCB & 2.00 & \num{0.0} & 5.33 & 3.90 & 0.94 & 0.10 \\
        auto-scaled DUCB & 1.50 & \num{-0.0001} & 5.00 & 3.67 & 0.91 & 0.16 \\
        PDUCB & 0.50 & \num{-1e-09} & 4.19 & 2.99 & 0.75 & 0.24 \\
        auto-scaled PDUCB & 1.00 & \num{-1e-09} & 4.10 & 2.85 & 0.76 & 0.24 \\
        GO & 0.10 & \num{0.0} & 5.40 & 3.88 & 0.95 & 0.09 \\
        \midrule
        \\
        \errtblhead{WRMISE}
        DUCB & 2.00 & \num{0.0} & 3.48 & 3.04 & 0.91 & 0.22 \\
        auto-scaled DUCB & 1.50 & \num{-0.0001} & 3.28 & 2.95 & 0.88 & 0.26 \\
        PDUCB & 0.50 & \num{-1e-05} & 2.29 & 2.20 & 0.63 & 0.39 \\
        auto-scaled PDUCB & 1.00 & \num{-1e-07} & 2.37 & 2.33 & 0.64 & 0.37 \\
        GO & 0.10 & \num{0.0} & 3.62 & 3.00 & 0.94 & 0.17 \\
    \end{errtbl}
    \caption[Minimal error values D-NF-SS-SV.]{The minimal obtained error (RMISE 
        and WRMISE) for each acquisition function and the parameter values used 
        in the single source Gaussian dispersion scenario 
        (D-NF-SS-SV).}\label{tbl:err-d-nf-ss-sv}
\end{table}

\begin{table}
    \centering
    \begin{errtbl}
        \errtblhead{RMISE}
        DUCB & 1.25 & \num{0.0} & 20.63 & 11.60 & 0.93 & 0.08 \\
        auto-scaled DUCB & 2.00 & \num{-0.0001} & 18.45 & 10.97 & 0.83 & 0.13 \\
        PDUCB & 0.50 & \num{0.0} & 16.55 & 11.76 & 0.68 & 0.18 \\
        auto-scaled PDUCB & 1.25 & \num{-1e-06} & 16.62 & 11.87 & 0.68 & 0.19 \\
        GO & 0.10 & \num{0.0} & 20.70 & 11.76 & 0.92 & 0.11 \\
        \midrule
        \\
        \errtblhead{WRMISE}
        DUCB & 0.50 & \num{-1e-05} & 14.11 & 10.03 & 0.93 & 0.14 \\
        auto-scaled DUCB & 2.00 & \num{-0.0001} & 12.11 & 9.60 & 0.80 & 0.21 \\
        PDUCB & 1.00 & \num{-1e-06} & 10.99 & 10.86 & 0.62 & 0.29 \\
        auto-scaled PDUCB & 1.25 & \num{-1e-05} & 10.80 & 10.59 & 0.65 & 0.26 \\
        GO & 0.10 & \num{0.0} & 14.06 & 9.96 & 0.93 & 0.12 \\
    \end{errtbl}
    \caption[Minimal error values D-NF-MS-SV.]{The minimal obtained error (RMISE 
        and WRMISE) for each acquisition function and the parameter values used 
        in the multiple source Gaussian dispersion scenario 
        (D-NF-MS-SV).}\label{tbl:err-d-nf-ms-sv}
\end{table}

\begin{figure}
    \centering
    \includegraphics{plots/errtrace-nf}
    \caption[Time-course of the error reduction]{The average remaining fraction 
        of the initial RMISE in the single source Gaussian scenario 
        (G-NF-SS-SV).  Each individual plot corresponds to one utility function 
        and scaling.  The auto-scaled versions use the scaling factor defined in 
        Equations~\ref{eqn:scale-ducb} and~\ref{eqn:scale-pducb}, in contrast to 
        a constant scaling factor. The $\gamma$ parameter is coded by hue and 
        the $\kappa$ parameter by lightness.}\label{fig:errtrace-nf}
\end{figure}

This is a rich dataset from which quite a few insights can be gained. First of 
all it can be noted that the GO acquisition function does not perform very well.  
Even in the single source Gaussian scenario the RMISE is only reduced by about 
\SI{20}{\percent} and in the other two scenarios it performs even worse.

Comparing DUCB and PDUCB the latter one consistently performs better with 
a reduction in the RMISE and WRMISE by at least additional \SI{11}{\percent} and 
in the dispersion scenarios even more.  Despite that, the standard deviation of 
PDUCB is higher almost always higher than that of DUCB\@.

In the single source Gaussian scenario PDUCB proves to be quite robust against 
the choice of $\kappa$ as for all values very good results are obtained. This 
picture is a bit more noisy in the dispersion scenarios. It seems that too low 
values ($\kappa < 0.5$) degrade performance. This is consistent with the 
argument in Chapter~\ref{sec:utility} that a too low $\kappa$ limits the 
exploration and let the UAV become stuck in a (local) maximum. The choice of 
$\gamma$ has no considerable effect as long as the distance penalty is not 
chosen too large ($\gamma < -10^{-5}$).

The same behavior for the choice of $\gamma$ is also observed for DUCB in the 
single source Gaussian scenario. However, this utility function is far more 
sensitive to the choice of $\kappa$. Using the scaling $s\ped{DUCB}(\vc y) = 1$ 
the performance degrades setting $\kappa < 1$ and using the automatic scaling it 
degrades for $\kappa < 1.5$. In the dispersion scenarios the DUCB acquisition 
function does not perform well for any tested combination of parameter values.

Interestingly,  DUCB performs slightly better with the automatic scaling, 
whereas PDUCB performs slightly worse.

Finally, taking a look at the time course of error reduction multiple phases can 
be discovered where a reasonable reduction of the error occurs. About the first 
\SI{500}{\second} nearly no reduction occurs as in this phase the UAV only 
surround the area of interest. Then the error rapidly decreases in the next 
\SIrange{1000}{2000}{\second} until the decrease levels off and stays fairly 
constant for the rest of the simulation time.

These results show that PDUCB outperforms the DUCB and GO acquisition functions 
and in addition is quite robust against a non-optimal choice of parameters. The 
especially bad performance of the GO utility function is not too surprising at 
its intended use is to find a function maximum and not building a correct model 
of the function (respectively plume concentration).  DUCB works reasonable well 
for the simple case of a Gaussian distribution, but fails for the more localized 
dispersions.

The PDUCB performance might not seem to be too impressive in the dispersion 
scenarios, too. However, one has to keep in mind that even in 
Section~\ref{sec:bestkernel} with much more samples the RMISE could not be 
reduced to less than \SI{61}{\percent}.  Also, the qualitatively the plume is 
predicted at the correct location as Figure~TODO shows, despite some deviance of 
the exact concentration values.

Given this discussion I decided to limit further experiments to the PDUCB 
acquisition function as the other options seem to not a viable option (except 
maybe for the single source Gaussian). As smaller values of $\kappa$ seem to 
provide slightly better results, but it should not be below $1$ as argued, 
a value of $\kappa = 1.25$ was selected. The penalty distance seems to slightly 
improve the results up to \num{-1e-5}. However, to prevent to fall in the area 
where the performs then quickly decreases, it was set to a lower value of 
$\gamma = \num{-1e-7}$ which also good results. Both scaling approaches 
(constant or automatic) where used in further experiments as it is not clear 
from these results which one is better. Though, the automatic scaling is a bit 
worse in terms of error, it has one less parameter which would be set according 
to the range of concentration values.

TODO example run visualizations

\section{Evaluation in a Noisy Setting}\label{sec:noisy}
It is important to verify that the methods for plume modelling also work in 
a noisy environment as in the real world noise will necessarily occur. For that 
a standard deviation of the sensor noise of $\sigma\ped{noise} 
= \SI{1e-5}{\gram\per\meter\cubed}$ was assumed. As it should be possible to 
reliably estimate the noise level of the sensor, I considered this value as 
known. That allows to set the noise variance of the Gaussian process 
$\sigma\ped{n}^2 = \sigma\ped{noise}^2$. For a good prediction the ratio of 
$\sigma\ped{n}^2$ and the kernel variance $\sigma\ped{k}^2$ has to be chosen 
well. The process of doing that will be discussed in the next section before 
returning to the actual simulations.

\subsection{Choosing the Kernel Variance}
The method for determining the kernel variance was essentially the same as 
described in Section~\ref{sec:bestkernel} for determining the length scale. The 
points were it differs are that instead of the noiseless scenarios the single 
and multiple source plume dispersion scenarios with sensor noise (D-SN-SS-SV, 
D-SN-MS-SV) were used and only the Mat\'ern kernel with $\nu = 3/2$ was used.  
Instead of varying the length scale it was fixed to $\ell = \SI{5}{\meter}$ and 
the kernel variance $\sigma\ped{k}^2$ was varied from 
\SIrange{1e-12}{1e-3}{\gram\squared\per\meter\tothe{6}}.

The results are shown in Figure~TODO\@. The normalized error measures are 
highest for low values of $\sigma\ped{k}^2$ approaching 1. For $\sigma\ped{k}^2 
> \SI{1e-9}{\gram\squared\per\meter\tothe{6}}$ the WRMISE stays the same, but 
the RMISE and QRSim reward slightly increase. This is even less pronounced for 
multiple sources.

As all error measures do have their minimum at $\sigma\ped{k}^2 
= \SI{1e-9}{\gram\squared\per\meter\tothe{6}}$ or have almost reached it, this 
value has been used in the following.

\subsection{Simulation of the Scenarios Including Noise}
Using the kernel variance of $\sigma\ped{k}^2 
= \SI{1e-9}{\gram\squared\per\meter\tothe{6}}$ determined in the previous 
section the PDUCB method was evaluated in the single and multiple source plume 
dispersion scenario including sensor noise (D-SN-SS-SV, D-SN-MS-SV). The sensor 
noise variance and the noise variance of the Gaussian process were set to 
$\sigma\ped{noise}^2 = \sigma\ped{n}^2 
= \SI{1e-10}{\gram\squared\per\meter\tothe{6}}$.  As the change of 
$\sigma\ped{k}^2$ (previously set to 1) influences $\sigma^2(\vc x)$ the value 
of $\kappa$ has to be adjusted by the inverse factor. Hence, the PDUCB parameter 
setting in the following were $\kappa = \num{1.25e9}$ and $\gamma 
= \num{-1e-7}$.

In many instances surrounding the simulation area in just one height is not 
sufficient to discover the plume under the influence of noise. Thus, the 
complete search and the wind based search strategy described in 
Chapter~\ref{sec:bootstrapping} have been employed. The latter approach requires 
of course wind information which was read out from the QRSim simulator.

Apart from these points the same methods as in the noiseless case 
(Section~\ref{sec:cmputility}) were used including the same number of 20~trials.

The results for a single source are summarized in Figure~TODO and Table~TODO\@.  
All tested variants reduce the normalized RMISE to approximately \num{0.75} with 
a similar standard deviation around \num{0.23}. Using the wind based search the 
error starts to decrease earlier as less area has to be covered. Also the 
overall decrease seems to be a bit faster after the plume has bee discovered.

Looking at the WRMISE it turns out that the wind based search decreases the 
normalized error by about \SI{7}{\percent} additionally compared to the complete 
search. Thus, the wind based search is able to better approximate the actual 
plume without increasing the approximation error in other area.

With regard to the WRMISE the automatic scaling seems to perform a bit better 
(difference of \num{0.05} in the normalized error), but this is not the case for 
the RMISE\@.

TODO multiple sources
TODO discussion/conclusion

\section{Multiple UAVs}
Finally, the performance of the PDUCB acquisition function with the extension 
for multiples UAVs (Chapter~\ref{sec:multiple-uavs}) has been evaluated. For 
this exactly the same methods as in the previous section were used with 
exception of the scenario. This was replaced by the multiple UAV, multiple 
source plume dispersion scenario (D-SN-MS-MV). Also, a number of different $\rho 
\in TODO$ has been tested.

TODO

\chapter{Outlook on Time-varying Plumes}\label{sec:timevarying}
So far only distributions which are static over time have been discussed. This 
assumption might be violated in many instances. Unfortunately, time did not 
allow me to extend the proposed methods to time-varying plumes. Nevertheless, 
I provide some thoughts on how to do this.

The first thing to do is probably adding another input dimension to the Gaussian 
process representing time. This also requires adjusting the covariance function.  
As the temporal correlations might differ from the spatial correlations 
a product of a temporal and a spatial kernels would be a good start. Though that 
separability neglects potential spatio-temporal interdependencies, construction 
and hyper-parameter estimation is easier. Given wind with a strong 
directionality, a kernel modeling the spatio-temporal interdependencies becomes 
more important. Some work on separable as well as non-separable spatio-temporal 
covariance functions with application to environmental monitoring has been 
published by \textcite{Singh:2010wt}.

Diffusion and advection by wind are the two main factors leading to a change of 
the plume distribution over time. While diffusion is a rather slow process, 
advection can occur on shorter time scales. Thus, it might be especially 
beneficial to include wind effects in the covariance function.  
\Textcite{Reggente:2009ti,Lilienthal:2009ij} did this in another kernel based 
approach called Kernel DM+V/W algorithm.

A specific scenario with a time-varying plume distribution is suggested by 
\textcite{denardi2013rn}. Instead of having a constant plume dispersion like in 
the scenarios discussed, each source emits puffs in random intervals travelling 
with the wind.  Given a regular emission interval this could be modeled by using 
the value of a periodic function as time input or incorporating a periodic 
function into the covariance function.  However, given a random emission 
interval this probably does not work as the frequency of the periodic function 
would have to change.

Another problem to be solved is locating such a puff dispersion. Already for the 
static dispersion an extensive search is required in the beginning. Given a puff 
dispersion measuring a low concentration could mean that the location does not 
lie in the path of the puff dispersion, but also that the measurement was taken 
between to puffs. Unfortunately, this seems like an inherent problem which 
cannot be completely solved.

Finally, let me discuss two properties of Gaussian processes to take into 
account when modeling time-varying plume distributions. When using a zero mean 
prior (as it is usually done), the predicted concentration mean will decay to 
zero over time (assuming $k(t, t') \rightarrow 0$ for $\abs{t - t'} \rightarrow 
\infty$). However, in certain scenarios (like puff dispersions) it is likely to 
measure an increased concentration again at a location in the future if this was 
the case once before at that location. It might be a good idea to adjust the 
mean prior based on the measurements to prevent a decay of the mean prediction.  
Note that the covariance function remains unchanged and the uncertainty at that 
location will still increase with time indicating that the measurement should be 
repeated.

The performance in time-varying scenarios might be impaired by the property that 
Gaussian processes have no notion of directionality of time. This follows from 
the (required) symmetry of the covariance function which does not allow to 
differentiate between $t < t'$ and $t > t'$. Hence, a kernel might model the 
path along which a plume travels, but not the direction. Along that direction 
the plume is likely to broaden because of dispersion. Respectively it gets more 
concentrated in the opposite direction. Effects like this can not be modeled 
except for the overall statistics.

\chapter{Conclusion}
Gaussian processes have been used before in modelling of spatial data and 
environmental monitoring. However, previous approaches, namely DUCB, do not work 
well for plume dispersions as the simulation experiments have shown. Presumably, 
the steep concentration gradients and high locality are the main problem.

In this work DUCB has been adapted to PDUCB which could be shown to work 
reasonable well also for plume dispersions. Nevertheless, there is still room 
for improvement as the error is seldom reduced by more than TODO\SI{1}{\percent} 
on average. The PDUCB algorithm could successfully be extended to multiple UAVs 
to increase the speed of mapping the dispersion.

Besides the modelling of the plume distribution, localizing the dispersion at 
all is also an important problem. Noisy data does not allow to reliably estimate 
a concentration gradient. This requires to use a systematic search approach.  
Incorporating information of the wind direction speed up the search. Once 
a plume has been found PDUCB maps it quickly.

One problem that could not be solved in the scope of this thesis is an automatic 
selection of hyper-parameters based on the data. The usual approach of 
likelihood optimization clearly fails. Selecting hyper-parameters on a trial to 
trial basis would also allow a closer match of the prediction to the plume 
dispersion.

Besides that, the prediction quality might be improved if wind information is 
considered and included in the covariance function. Some pointer in that 
direction, though for a different algorithm, are given by 
\textcite{Reggente:2009ti}. In general non-stationary kernels could improve the 
prediction, but they come at the cost of more hyper-parameters and prior 
assumptions about the plume dispersion which might be violated.

In summary, the basic applicability of Gaussian processe with a PDUCB 
acquisition function for plume distribution modelling has been shown. However, 
there is quite a number of possibilities of improvement left for future work.  
Also, it should be shown in future work that the proposed methods work in a real 
world scenario as only simulations have been performed. Further interesting 
research directions following from this work would be the inclusion and handling 
of obstacles or the modelling of time-varying plume distributions.



\appendix
\chapter{Error Bound of a Mean Estimate}\label{sec:decnoise}
\begin{theorem}
    Let $y = y^* + \eta$ with $\eta \sim \mathcal N(0, \sigma^2)$ and $\bar y$ 
    be the mean of $n$ samples from $y$. Then $n \geq 1.96^2 \cdot \del{\sigma 
        / \rho}^2$ samples are needed to have the error bound $\abs{\bar{y} 
        - y^*} < \rho$ hold with probability $p \geq 0.95$.
\end{theorem}

\begin{proof}
    Given the error bound the true value $y^*$ has to, with probability $p$, lie 
    in $\intcc{\bar y - \rho, \bar y + \rho}$. Thus, the \SI{95}{\percent} 
    confidence interval of $\bar y$ has to be a subset of that. With the 
    standard error $\sigma / \sqrt{n}$ the confidence interval is obtained as
    \begin{equation*}
        y^* \in \intcc{\bar y - 1.96 \cdot \frac{\sigma}{\sqrt{n}}, \bar{y} 
            + 1.96 \cdot \frac{\sigma}{\sqrt{n}}} \subseteq \intcc{\bar{y} 
            - \rho, \bar y + \rho} \text{.}
    \end{equation*}
    From that follows
    \begin{equation*}
        \rho \geq 1.96 \cdot \frac{\sigma}{\sqrt{n}} \quad\Leftrightarrow\quad 
        n \geq 
        1.96^2 \del{\frac{\sigma}{\rho}}^2
    \end{equation*}
\end{proof}

Assuming a Gaussian plume dispersion (Equation~\ref{eqn:plumedisp}) with the 
highest concentration possible in the scenarios $Q 
= \SI{2.5}{\gram\per\second}$, and
$u = \SI{3}{\meter\per\second}, \vc s' = (0, 0, \SI{-40}{\meter})\Tr$ 
a concentration of $c(\vc x') \approx \SI{0.055}{\gram\per\meter\cubed}$ is 
obtained at $\vc x' = (\SI{10}{\meter}, 0, \SI{-40}{\meter})\Tr$, a position in 
the center of the plume ten meters away from the source. For a usable 
measurement the magnitude of noise should be at least a magnitude lower, thus 
$\rho \leq \SI{0.0055}{\gram\per\meter\cubed}$. From the theorem it follows 
that, given the QRSim default noise standard deviation of $\sigma\ped{sn} 
= \SI{e-2}{\gram\per\meter\cubed}$ at least 385~samples are needed. Further away 
from the source or with a lower emission rate (which can also be a magnitude 
lower) even more samples would be needed.

\chapter{Sparse Online Gaussian Processes}\label{sec:sparse-gp-apdx}
In the following it will be proven that the matrix $-\mat C_t$ of a sparse 
online Gaussian process \parencite{Csato:2002fp} is symmetric, 
positive-definite. The proof allows to relate the rule for full updates to the 
update of the inverse Cholesky factor in Chapter~\ref{sec:onlineup}. As the 
notation by \textcite{Csato:2002fp} differs it should be noted that 
$[\sigma_x^2] = \mat B$ and $\vc k_{t+1} = K(X, {\vc x_{t+1}})$.

\begin{theorem}
    The matrix $-\mat C_t$ is symmetric, positive-definite for all $t \geq 1$.
\end{theorem}

\begin{proof}
    The proof is done by induction. It has to be shown
    \begin{itemize}
        \item that $-\mat C_1$ (base case) fulfills the proposition
        \item and that $-\mat C_{t+1}$ fulfills the proposition given it is 
            fulfilled for $\mat C_t$ (inductive step).
    \end{itemize}
    The deletion of a basis vector has not to be considered as it exactly undoes 
    a full update and then performs a reduced update.

    \paragraph{Base Case}
    For $t = 1$ we obtain
    \begin{equation*}
        -\mat C_1 = \sbr{r^{(t+1)}} = \sbr{\sigma_x^{-2}}
    \end{equation*}
    As $\sigma_x > 0$ it follows that $-\mat C_1$ is symmetric positive-definite.

    \paragraph{Inductive Step}
    For showing the symmetry and positive-definiteness after a full update it 
    suffices to show that Cholesky factorization for the updated matrix $-\mat 
    C_{t+1} = \del[0]{\mat L'^{-1}}\Tr \mat L'^{-1}$ exists.
\begin{align*}
    -\mat C_{t+1} &= -U_{t+1}(\mat C_t) - r^{(t+1)} \vc s_{t+1} \vc s_{t+1}\Tr 
    \\
        &= \sbr{\begin{array}{cc}
                \mat -C_t + \sigma_x^{-2} \mat C_t \vc k_{t+1} \vc k_{t+1}\Tr 
                \mat C_t\Tr & \sigma_x^{-2} \mat C_t \vc k_{t+1} \vc\e_{t+1}\Tr 
                \\[\smallskipamount]
                \sigma_x^{-2} \vc\e_{t+1} k_{t+1}\Tr \mat C_t\Tr & \sigma_x^{-2}
            \end{array}} \\
        &= \sbr{\begin{array}{cc}
                \del[0]{\mat L^{-1}}\Tr & \sigma_x^{-1} \mat C_t \vc k_{t+1} \\
                0 & \sigma_x^{-1}
            \end{array}} \sbr{\begin{array}{cc}
                \mat L^{-1} & 0 \\
                \sigma_x^{-1} \vc k_{t+1}\Tr \mat C_t\Tr & \sigma_x^{-1}
            \end{array}} \\
        &= \del[0]{\mat L'^{-1}}\Tr \mat L'^{-1}
    \end{align*}

    In case of a reduced update the relation
    \begin{equation*}
        -\mat C_{t+1} = -\mat C_t - r^{(t+1)} \vc s_{t+1} \vc s_{t+1}\Tr
    \end{equation*}
    holds. The term $- r^{(t+1)} \vc s_{t+1} \vc s_{t+1}\Tr$ is symmetric, 
    positive-definite as it is an outer vector product (which is symmetric, 
    positive-definite) multiplied by a positive number $-r^{(t+1)} 
    = \sigma_x^{-2} > 0$. The sum $-\mat C_{t+1}$ of symmetric, 
    positive-definite matrices is again symmetric, positive-definite.
\end{proof}

\begin{corollary}
    A full update of a symmetric, positive-definite matrix $-\mat{C}_t$ as 
    formulated by \textcite[equation~2.9]{Csato:2002fp} consists of the same 
    calculations as an online update of the inverse Cholesky factor in 
    Equation~\ref{eqn:invChol}.
\end{corollary}
This is obvious from the inductive step for a full update.

\chapter{PDUCB Differentiability}\label{sec:pducb-diff}
\begin{theorem}
    The PDUCB acquisition function given by
    \begin{align*}
    u\ped{PDUCB}(\vc x) &= u_1(\vc x) + s\ped{PDUCB}(\vc y) u_2(\vc x) \\
    u_1(\vc x) &= \del{1 - a} \cdot \ln\del{\mu_+(\vc x) + \varepsilon} 
    + a \cdot \ln \varepsilon \\
    u_2(\vc x) &= \kappa \cdot \del[1]{\sigma^2(\vc x) - \sigma\ped{n}^2} 
    + \gamma \cdot d^2(\vc x, \vc x')
    \end{align*}
    with
    \begin{align*}
    a &= \e^{-\mu_+(\vc x) / \varepsilon} \\
    \mu_+(\vc x) &= \max\cbr{0, \mu(\vc x)}
    \end{align*}
    is differentiable by $\vc x$ for all $\vc x$ if the Gaussian process 
    providing $\mu(\vc x)$ and $\sigma^2(\vc x)$ is mean square differentiable.
\end{theorem}

\newcommand{\dx}{\od{}{\vc{x}}}
\begin{proof}
    As
    \begin{equation*}
        \dx u\ped{PDUCB}(\vc x) = \dx u_1(\vc x) + s\ped{PDUCB}(\vc y) \dx 
        u_2(\vc x)
    \end{equation*}
    it suffices to show the differentiability for $u_1(\vc x)$ and $u_2(\vc x)$ 
    independently.

    For $u_1(\vc x)$ the derivative is
    \begin{align*}
        %\dx u_1(\vc x) &= \dx\!\del{1 - \e^{-\mu_+(\vc x)/\varepsilon}} \cdot 
        %\ln\!\del{\mu_+(\vc x) + \varepsilon} \\
        %&\quad + \del{1 - \e^{-\mu_+(\vc x)/\varepsilon}} \dx\!\del{\mu_+(\vc 
            %x)} \frac{1}{\mu_+(\vc x) + \varepsilon} \\
        %&\quad + \ln \varepsilon \cdot \dx \e^{-\mu_+(\vc x) / \varepsilon} \\
        \dx u_1(\vc x) &= \dx\!\del{1 - a} \cdot \ln\!\del{\mu_+(\vc x) 
            + \varepsilon} \\
        &\quad + \del{1 - a} \dx\!\del{\mu_+(\vc x)} \frac{1}{\mu_+(\vc x) 
            + \varepsilon} \\
        &\quad + \ln \varepsilon \cdot \dx a \\
        %&= \dx\!\del{\mu_+(\vc x)} \varepsilon^{-1} \e^{-\mu_+(\vc 
            %x)/\varepsilon} \cdot \ln\!\del{\mu_+(\vc x) + \varepsilon} \\
        %&\quad + \del{1 - \e^{-\mu_+(\vc x)/\varepsilon}} \dx\!\del{\mu_+(\vc 
            %x)} \frac{1}{\mu_+(\vc x) + \varepsilon} \\
        %&\quad + \ln \varepsilon \cdot \dx\!\del{\mu_+(\vc x)} \varepsilon^{-1} 
        %\e^{-\mu_+(\vc x) / \varepsilon} \\
        &= \dx\!\del{\mu_+(\vc x)} a\varepsilon^{-1} \cdot \ln\!\del{\mu_+(\vc 
            x) + \varepsilon} \\
        &\quad + \del{1 - a} \dx\!\del{\mu_+(\vc x)} \frac{1}{\mu_+(\vc x) 
            + \varepsilon} \\
        &\quad - \ln \varepsilon \cdot \dx\!\del{\mu_+(\vc x)} a\varepsilon^{-1}
        \\
        &= \dx\!\del{\mu_+(\vc x)} \cdot \del{a\varepsilon^{-1} 
            \ln\!\del{\mu_+(\vc x) + \varepsilon} - a\varepsilon^{-1} \ln 
            \varepsilon + \frac{1 - a}{\mu_+(\vc{x}) + \varepsilon}} \\
        &= \left\{ \begin{array}{ll}
                \dx\!\del{\mu(\vc x)} \cdot \del{a\varepsilon^{-1} 
                    \ln\!\del{\mu(\vc x) + \varepsilon} - a\varepsilon^{-1} \ln 
                    \varepsilon + \frac{1 - a}{\mu(\vc{x}) + \varepsilon}} 
                & \mu(\vc x) > 0 \\
                \dx\!\del{0} \cdot \del{\varepsilon^{-1} \ln \varepsilon 
                    - \varepsilon^{-1} \ln \varepsilon + \frac{1 
                        - 1}{\varepsilon}} = 0 & \mu(\vc x) \leq 0
            \end{array} \right. \text{.}
        \end{align*}
    Hence, $u_1(\vc x)$ is always differentiable for all $\vc x$ with $\mu(\vc 
    x) \leq 0$. Furthermore, the parenthesized term is a composition of 
    continuous functions for $\mu(\vc x) > 0$ and the derivative $\dx \mu(\vc 
    x)$ exists for a mean square differentiable Gaussian process. From that, the 
    differentiability for all $\vc x$ follows.

    For $u_2(\vc x)$ the somewhat simpler derivative is
    \begin{equation*}
        u_2(\vc x) = \kappa \cdot \dx \sigma^2(\vc x) + \gamma \cdot 
        \dx\!\del{d^2(\vc x, \vc x')}
    \end{equation*}
    in which $\dx \sigma^2(\vc x)$ is differentiable for a mean square 
    differentiable Gaussian process and the distance derivative is given by
    \begin{equation*}
        \dx d^2(\vc x, \vc x') = 2 (\vc x - \vc x') \text{.}
    \end{equation*}
    \end{proof}

\chapter{Prior Width}\label{sec:prior}
\begin{theorem}
    To increase the log likelihood $\ln p(\vc y | X, \vc \theta, \mathcal{H}_j)$ 
    by at least $\ln(\Delta p)$ at $\theta_i = m_{\theta_i}$ with a Gaussian 
    prior at $m_{\theta_i}$ the width $\sigma_{\theta_i}$ of this prior has to 
    be lower or equal than $1 / (\Delta p \sqrt{2\uppi})$.
\end{theorem}

\begin{proof}
    The log likelihood combined with (independent) priors is given by
    \begin{equation*}
        \ln\!\del{p(\vc y | X, \vc \theta, \mathcal{H}_j) p(\vc \theta 
            | \mathcal{H}_j)} = \ln p(\vc y | X, \vc \theta, \mathcal{H}_j) 
        + \sum_{i = 1}^n \ln p(\theta_i | \mathcal{H}_j) \text{.}
    \end{equation*}
    Hence, the condition
    \begin{equation*}
        \ln p(\theta_i | \mathcal{H}_j) \geq \ln(\Delta p)
    \end{equation*}
    has to be fulfilled. Inserting the Gaussian prior it becomes:
    \begin{align*}
        &N(m_{\theta_i}; m_{\theta_i}, \sigma_{\theta_i}) 
        = \ln\!\del{\frac{1}{\sigma_{\theta_i} \sqrt{2\uppi}}} \geq \ln(\Delta 
        p) \\
        \Leftrightarrow\quad & \frac{1}{\Delta p\sqrt{2\uppi}} \geq 
        \sigma_{\theta_i}
    \end{align*}
\end{proof}

In Chapter~\ref{sec:bestkernel} the difference of the maximum likelihood and the 
likelihood at $\ell = \SI{5}{\meter}$ is $\ln(\Delta p) \approx 2078$ in the 
G-NF-SS scenario. To shift the likelihood maximum to $\ell = \SI{5}{\meter}$ 
with a Gaussian prior with its mean at that position the width would need to be 
less than $\e^{-2078} / \sqrt{2\uppi} \approx 0$.


\backmatter{}
\listoffigures{}
\listoftables{}
\printbibliography{}

\end{document}
